% \iffalse meta-comment
%
% rustic.dtx
% Copyright 1999 Peter R. Wilson
%
% This program is provided under the terms of the
% LaTeX Project Public License distributed from CTAN
% archives in directory macros/latex/base/lppl.txt.
%
% Author: Peter Wilson (CUA) (now at: peter.r.wilson@boeing.com)
% 
% If you do not have the docmfp package (available from CTAN in
% tex-archive/macros/latex/contrib/supported), comment out the
% \usepackage{docmfp} line below and uncomment the line following it. 
% 
%<*driver>
\documentclass{ltxdoc}
\usepackage{docmfp}
%%%%%% \providecommand{\DescribeVariable}[1]{} \newenvironment{routine}[1]{}{}
\EnableCrossrefs
\CodelineIndex
\setcounter{StandardModuleDepth}{1}
\begin{document}
  \DocInput{rustic.dtx}
\end{document}
%</driver>
%
% \fi
%
% \CheckSum{137}
%
% \DoNotIndex{\',\.,\@M,\@@input,\@addtoreset,\@arabic,\@badmath}
% \DoNotIndex{\@centercr,\@cite}
% \DoNotIndex{\@dotsep,\@empty,\@float,\@gobble,\@gobbletwo,\@ignoretrue}
% \DoNotIndex{\@input,\@ixpt,\@m}
% \DoNotIndex{\@minus,\@mkboth,\@ne,\@nil,\@nomath,\@plus,\@set@topoint}
% \DoNotIndex{\@tempboxa,\@tempcnta,\@tempdima,\@tempdimb}
% \DoNotIndex{\@tempswafalse,\@tempswatrue,\@viipt,\@viiipt,\@vipt}
% \DoNotIndex{\@vpt,\@warning,\@xiipt,\@xipt,\@xivpt,\@xpt,\@xviipt}
% \DoNotIndex{\@xxpt,\@xxvpt,\\,\ ,\addpenalty,\addtolength,\addvspace}
% \DoNotIndex{\advance,\Alph,\alph}
% \DoNotIndex{\arabic,\ast,\begin,\begingroup,\bfseries,\bgroup,\box}
% \DoNotIndex{\bullet}
% \DoNotIndex{\cdot,\cite,\CodelineIndex,\cr,\day,\DeclareOption}
% \DoNotIndex{\def,\DisableCrossrefs,\divide,\DocInput,\documentclass}
% \DoNotIndex{\DoNotIndex,\egroup,\ifdim,\else,\fi,\em,\endtrivlist}
% \DoNotIndex{\EnableCrossrefs,\end,\end@dblfloat,\end@float,\endgroup}
% \DoNotIndex{\endlist,\everycr,\everypar,\ExecuteOptions,\expandafter}
% \DoNotIndex{\fbox}
% \DoNotIndex{\filedate,\filename,\fileversion,\fontsize,\framebox,\gdef}
% \DoNotIndex{\global,\halign,\hangindent,\hbox,\hfil,\hfill,\hrule}
% \DoNotIndex{\hsize,\hskip,\hspace,\hss,\if@tempswa,\ifcase,\or,\fi,\fi}
% \DoNotIndex{\ifhmode,\ifvmode,\ifnum,\iftrue,\ifx,\fi,\fi,\fi,\fi,\fi}
% \DoNotIndex{\input}
% \DoNotIndex{\jobname,\kern,\leavevmode,\let,\leftmark}
% \DoNotIndex{\list,\llap,\long,\m@ne,\m@th,\mark,\markboth,\markright}
% \DoNotIndex{\month,\newcommand,\newcounter,\newenvironment}
% \DoNotIndex{\NeedsTeXFormat,\newdimen}
% \DoNotIndex{\newlength,\newpage,\nobreak,\noindent,\null,\number}
% \DoNotIndex{\numberline,\OldMakeindex,\OnlyDescription,\p@}
% \DoNotIndex{\pagestyle,\par,\paragraph,\paragraphmark,\parfillskip}
% \DoNotIndex{\penalty,\PrintChanges,\PrintIndex,\ProcessOptions}
% \DoNotIndex{\protect,\ProvidesClass,\raggedbottom,\raggedright}
% \DoNotIndex{\refstepcounter,\relax,\renewcommand,\reset@font}
% \DoNotIndex{\rightmargin,\rightmark,\rightskip,\rlap,\rmfamily,\roman}
% \DoNotIndex{\roman,\secdef,\selectfont,\setbox,\setcounter,\setlength}
% \DoNotIndex{\settowidth,\sfcode,\skip,\sloppy,\slshape,\space}
% \DoNotIndex{\symbol,\the,\trivlist,\typeout,\tw@,\undefined,\uppercase}
% \DoNotIndex{\usecounter,\usefont,\usepackage,\vfil,\vfill,\viiipt}
% \DoNotIndex{\viipt,\vipt,\vskip,\vspace}
% \DoNotIndex{\wd,\xiipt,\year,\z@}
%
% \changes{v1.0}{1999/05/01}{First public release}
% \changes{v1.0a}{2001/01/02}{Fixed missing * in .fd files}
%
% ^^A \def\fileversion{v1.0}
% ^^A \def\filedate{1999/05/01}
% \def\fileversion{v1.0a}
% \def\filedate{2001/01/02}
% \newcommand*{\Lpack}[1]{\textsf {#1}}           ^^A typeset a package
% \newcommand*{\Lopt}[1]{\textsf {#1}}            ^^A typeset an option
% \newcommand*{\file}[1]{\texttt {#1}}            ^^A typeset a file
% \newcommand*{\Lcount}[1]{\textsl {\small#1}}    ^^A typeset a counter
% \newcommand*{\pstyle}[1]{\textsl {#1}}          ^^A typeset a pagestyle
% \newcommand*{\Lenv}[1]{\texttt {#1}}            ^^A typeset an environment
% \newcommand*{\AD}{\textsc{ad}}
%
% \title{The \Lpack{rustic} fonts\thanks{This
%        file has version number \fileversion, last revised
%        \filedate.}}
%
% \author{%
% Peter Wilson\\
% Catholic University of America \\
% Now at \texttt{peter.r.wilson@boeing.com}
% }
% \date{\filedate}
% \maketitle
% \begin{abstract}
%    The \Lpack{rustic} and \Lpack{allrust} packages provide a set of 
% Metafont Roman Rustic book-hands 
% as used
% for manuscripts in the 1st to the 6th century. This is one in a series
% of manuscript fonts.
% \end{abstract}
% \tableofcontents
%
% \StopEventually{}
%
% 
%
% \section{Introduction}
%
%    The \Lpack{rustic} and \Lpack{allrust} packages provide 
% a Metafont~\cite{METAFONT} version
% of the Roman Rustic manuscript book-hand that was in use between about the
% first and sixth centuries~\AD. It is part of a project to provide fonts
% covering the major manuscript hands between the first century~\AD{} and
% the invention of printing. The principal resources used in this project
% are listed in the Bibliography.
%
% This manual is typeset according to the conventions of the
% \LaTeX{} \textsc{docstrip} utility which enables the automatic
% extraction of the \LaTeX{} macro source files~\cite{GOOSSENS94}.
% The \Lpack{docmfp} package is used for documenting the Metafont portions
% of the distribution~\cite{DOCMFP}.
%
%    Section~\ref{sec:usc} describes the usage of the package.
% Commented code for the fonts is in Sections~\ref{sec:mf} and~\ref{sec:fd}
% and source code for the packages is in Section~\ref{sec:code}.
%
% \subsection{Manuscript book-hands}
%
% Before the invention of printing all books were written by hand. The book-hands
% used by the scribes and copyists for the manuscripts changed as time 
% went on. Table~\ref{tab:1} lists some of the common book-hands which were used
% between the 1st and 15th centuries. The later book-hands formed the basis of the
% fonts used by the early printers, which in turn form the basis of the printing
% fonts in use today.
%
%    The manuscript book-hands were written with a broad nibbed reed or quill
% pen. Among the distinguishing characteristics of a hand, apart from the 
% actual shape of the letters, are the angle of the pen (which controls the
% variation between thick and thin strokes) and the height of a letter compared
% to the width of the nib. The lower the ratio of the letter height to nib
% width, the more condensed is the script. The scripts also varied in their
% typical height. 
%
%    Table~\ref{tab:1} gives an `average' x-height for each
% script, which I obtained by measuring a sample of photographs of
% manuscripts written in the various hands. About a dozen examples
% of each book-hand were measured. This figure should not be taken too
% seriously.
%
%    There was not a sharp division between the use of one hand and another.
% Many manuscripts exhibit a variety of hands in the same document. For example,
% the scribe writing in an Uncial hand may have used Roman Rustic letters
% for capitals. Usually the same pen was used for the two different scripts.
%
%    Generally speaking, as a hand got older it became more embelished and
% calligraphic, and 
% therefore took longer to write. As this happened a new hand would appear that
% was faster, and which would eventually make the earlier one obsolete.
%
%    Many of the book-hands were single-cased; that is, they did not have an
% upper- and lower-case as we do nowadays in Western scripts. On the other
% hand, a script might be majuscule or minuscule. A \textit{majuscule} script
% is one, like our upper-case, where the letters are drawn between two lines
% and are of a uniform height with no ascenders or descenders. 
% A \textit{minuscule} script, like our lower-case, is drawn between four lines 
% and has ascenders and  descenders.
%
% \begin{table}
% \centering
% \caption{The main manuscript book-hands} \label{tab:1}
% \begin{tabular}{lccccc} \hline
% Name                  & Century  & x-height & Height     & Pen  & Normalised \\
%                       &          & (mm)   & (nib widths) & angle & height  \\ \hline
% Roman Rustic          & 1--6     & 5.7    & 4--6         & 45     & 1.90 \\
% Uncial                & 3--6     & 4.1    & 4--5         & 30     & 1.37 \\
% Artificial Uncial     & 6--10    & 4.2    & 3--6         & 10     & 1.40 \\
% Half Uncial           & 3--9     & 3.2    & 3--6         & 20--30 & 1.07 \\
% Insular majuscule     & 6--9     & 4.2    & 5            & 0--20  & 1.40 \\
% Insular minuscule     & 6 onward & 4.1    & 5--6         & 45--70 & 1.37 \\
% Carolingian minuscule & 8--12    & 3.0    & 3--5         & 45     & 1.00 \\
% Early Gothic          & 11--12   & 3.8    & 4--6         & 20--45 & 1.27 \\
% Gothic Textura        & 13--15   & 3.9    & 3--5         & 30--45 & 1.30 \\
% Gothic Prescius       & 13 onward & 3.3   & 4--5         & 45     & 1.10 \\
% Rotunda               & 13--15   & 3.2    & 4--6         & 30     & 1.07 \\
% Humanist minuscule    & 14 onward & 3.0   & 4--5         & 30--40 & 1.00 \\ \hline
% \end{tabular}
% \end{table}
%
%    During the period under consideration arabic numerals were effectively
% not used. At the beginning they were unknown and even though some knew of
% them towards the end, the glyphs used for them are not recognisable ---
% to me they look somewhat like cryllic letters --- and each locality and time
% had its own highly individualistic rendering. The general rule when using
% one of these book-hands is to write all numbers using roman numerals.
%
%    In the first century punctuation was virtually unknown, and typically there
% would not even be any additional space between individual words, never mind
% denoting ends of sentences. Sometimes a dot at mid-height would be used as
% a word seperator, or to mark off the end of a paragraph. Effectively a text
% was a continuous stream of letters. By the time that printing was invented,
% though, all of our modern punctuation marks were being used.
%
%    Among all these manuscript hands, the Carolingian minuscule is the
% most important as our modern fonts are based on its letter shapes, and it is
% also at this point in time where the division occured between the black letter
% scripts as used even today in Germany, and the lighter fonts used elesewhere.
% The Rotunda and Humanist minuscule hands were developed in Italy and were
% essentially a rediscovery of the Carolingian minusucle. Guthenberg took the
% Gothic scripts as the model for his types. Later printers, 
% such as Nicholas Jenson of Venice,
% took the Humanist scripts as their models.
%
%
% \subsection{The Roman Rustic script}
%
%    The Roman Rustic hand, which is a majuscule script, was in use for some five 
% centuries until it was suceeded by the Uncial book-hands. Usually the
% lettering in a manuscript was all one size. If the scribe felt the need
% for `capital' letters then, using the same pen, would write a slightly larger
% letter.
% During the period when the
% script was used the alphabet only had 23 letters. I have included
% the missing J, U and W.
%
%    Arabic numerals were unknown at this time, so all numbers were written
% using the roman numbering system. I have provided Roman Rustic versions of
% the arabic digits.
%
%    Punctuation was non-existant. A dot at mid-height might be used as
% a word seperator, but that was all. Paragraphs had no indentation, neither
% was there any extra vertical space between paragraphs. Essentially, the
% script ran continuously with no spaces, even between words.
%
%    The `leading' --- the amount of white space between lines --- was about
% the same height as the letters.
%
%
% \section{The \Lpack{rustic} and \Lpack{allrust} package} \label{sec:usc}
%
%     The Roman Rustic font family is called |rust|. The font is supplied in both
% OT1 and T1 encodings.
%
% \subsection{The \Lpack{rustic} package}
%
% This is intended for the occasions when some short pieces of text have to be
% written in Roman Rustic while the majority of the document is in another font.
% The normal baselineskips are used.
%
% \DescribeMacro{\rustfamily}
%    The |\rustfamily| declaration starts typesetting with the Roman Rustic fonts.
% Use of the Rustic font will continue until either there is another |\...family|
% declaration or the current group (e.g., environment) is closed.
%
% \DescribeMacro{\textrust}
%    The command |\textrust{|\meta{text}|}| will typeset \meta{text} using the
% Roman Rustic fonts.
%
% \subsection{The \Lpack{allrust} package}
%
%    This package is for when the entire document will be typeset with the
% Roman Rustic font. The baselineskips are set to those appropriate to the
% book-hand. 
%
%    This is a minimalist package. Apart from declaring Roman Rustic to be the
% default font and setting the baselineskips appropriately, it makes no other
% alterations. 
% Vertical spacing
% before and after section titles and before and after lists, etc., will be
% too small as the \LaTeX{} design assumes a font comparable in size to
% normal printing fonts, and the book-hand is much taller.
% To capture more of the flavour of the time, all numbers
% should be written using roman numerals. 
% The \Lpack{romannum} package~\cite{ROMANNUM}
% can be used so that \LaTeX{} will typeset the numbers that it generates,
% like sectioning or caption numbers, using roman numerals instead of arabic 
% digits. 
%
% \DescribeMacro{\cmrfamily}
% \DescribeMacro{\textcmr}
% \DescribeMacro{\cmssfamily}
% \DescribeMacro{\textcmss}
% \DescribeMacro{\cmttfamily}
% \DescribeMacro{\textcmtt}
%    The |...family| declarations start typesetting with the Computer Modern Roman
% (|\cmrfamily|), the Computer Modern Sans (|\cmssfamily|), and the Computer
% Modern Typewriter (|\cmttfamily|) font families. The |\textcm..{|\meta{text}|}|
% commands will typeset \meta{text} in the corresponding Computer Modern font.
%
%    The \Lpack{allrust} package automatically loads the \Lpack{rustic} package,
% so the rustic font commands are available if necessary.
%
%
%
% \section{The Metafont code} \label{sec:mf}
%
%
%    As previously noted, this work is part of a larger project to provide
% fonts covering the main manuscript book-hands. As such, one of the
% aims is to produce a coordinated set of fonts, especially as multiple
% hands might be used in a single document. 
%
%    Noting that the hands tend to be somewhat larger than the typical 10pt
% size (where the x-height is approximately 1.5mm) used for modern books, 
% I have also designed the fonts
% at a larger than normal size, then applied some non-linear factors when reducing
% them down to a 10pt size.
%    Further, I have used the height of the Carolingian minuscule as a 
% normalising factor when deciding on the absolute height of any particular
% script. The x-height of the Carolingian font is made equal to the
% x-height of the Computer Modern Roman (CMR) font.
%
%    Modern fonts are effectively drawn. That is, the outline of the letter is
% drawn carefully and the center is filled with ink. This is shown to good
% effect in the Metafont code for the Computer Modern fonts~\cite{CM}.
% In contrast, the manuscript letters were inked by single pen strokes in
% a calligraphic manner. I have tried to repeat this calligraphic style
% in the Metafont code.
%
%    As much as possible I have tried to use parameter values from the
% Computer Modern Roman (CMR) fonts in order to reduce possible
% infelicities if the CM and manuscript fonts are used together. However,
% few of the CMR parameters are applicable to the calligraphic style.
%
%
% \subsection{The parameter file}
%
%    We deal with the parameter files first, and start by announcing
% what they are for. The Rustic font comes in three sizes and also as a normal
% and a bold font.
%    \begin{macrocode}
%<*base7|base10|base17|base7b|base10b|base17b>
%<base7>%%% RUST7.MF    Roman Rustic at 7 point design size.
%<base10>%%% RUST10.MF  Roman Rustic at 10 point design size.
%<base17>%%% RUST17.MF  Roman Rustic at 17 point design size.
%<base7b>%%% RUSTB7.MF    Roman Rustic Bold at 7 point design size.
%<base10b>%%% RUSTB10.MF  Roman Rustic Bold at 10 point design size.
%<base17b>%%% RUSTB17.MF  Roman Rustic Bold at 17 point design size.
%
%    \end{macrocode}
%    Parameters from CMR are used as much as possible. We also
% make sure that \Lpack{cmbase} is loaded as well as plain Metafont.
%    \begin{macrocode}
if unknown cmbase: input cmbase fi

%<base7>font_identifier:="RUST"; font_size 7pt#;
%<base10>font_identifier:="RUST"; font_size 10pt#;
%<base17>font_identifier:="RUST"; font_size 17.28pt#;
%<base7b>font_identifier:="RUSTB"; font_size 7pt#;
%<base10b>font_identifier:="RUSTB"; font_size 10pt#;
%<base17b>font_identifier:="RUSTB"; font_size 17.28pt#;

%    \end{macrocode}
%
% \DescribeVariable{jutstretch}
%    The scaling for minuscule serifs wrt 17pt size.
%    \begin{macrocode}
%<base7|base7b> jutstretch:=1.19; 
%<base10|base10b> jutstretch:=1.152; 
%<base17|base17b> jutstretch:=1.0;
%    \end{macrocode}
%
% \DescribeVariable{stemstretch}
%    The scaling for minuscule stem widths wrt 17pt size.
%    \begin{macrocode}
%<base7|base7b> stemstretch:=1.50;
%<base10|base10b> stemstretch:=1.31;
%<base17|base17b> stemstretch:=1.0;
%    \end{macrocode}
%
% \DescribeVariable{caprat}
%    The scaling for `capitals' wrt `lowercase'. This is pretty much a guess.
%    \begin{macrocode}
 caprat:=1.2;          % ratio of capital height to minuscule height
%    \end{macrocode}
%
% \DescribeVariable{cap_jutstretch}
%    Scaling for majuscule serifs wrt 17pt size.
%    \begin{macrocode}
%<base7|base7b> cap_jutstretch:=1.3;   
%<base10|base10b> cap_jutstretch:=1.2;   
%<base17|base17b> cap_jutstretch:=1.0;   
%    \end{macrocode}
%
% \DescribeVariable{cap_stemstretch}
%    Scaling for majuscule stem width wrt 17pt size.
%    \begin{macrocode}
%<base7|base7b> cap_stemstretch:=1.45;  
%<base10|base10b> cap_stemstretch:=1.31;  
%<base17|base17b> cap_stemstretch:=1.0;  
%    \end{macrocode}
%
% \DescribeVariable{bfudge}
%    Letter width scaling for bold font wrt normal font.
%    \begin{macrocode}
%<base7|base10|base17> bfudge:=1.0;
%<base7b|base10b|base17b> bfudge:=1.2;
%    \end{macrocode}
%
% \DescribeVariable{szfudge}
%    Width scaling wrt 17pt size letter width.
%    \begin{macrocode}
%<base7|base7b> szfudge:=1.18; 
%<base10|base10b> szfudge:=1.0; 
%<base17|base17b> szfudge:=1.0; 
%    \end{macrocode}
%
% \DescribeVariable{hstretch}
%    Horizontal stretching factor.
%    \begin{macrocode}
 hstretch:=szfudge*bfudge;
%    \end{macrocode}
%
% \DescribeVariable{carol_height}
%    The x-height of the Carolingian minuscule font.
%    \begin{macrocode}
%<base7|base7b> carol_height#:=108.5/36pt#; 
%<base10|base10b> carol_height#:=155/36pt#; 
%<base17|base17b> carol_height#:=268/36pt#; 
%    \end{macrocode}
%
% \DescribeVariable{vstretch}
%    The height of this font wrt the Carolingian font.
%    \begin{macrocode}
 vstretch:=1.9;   

%    \end{macrocode}
%
% \DescribeVariable{x_height} 
%    The x-height of lower case letters. Scaled from the Carolingian font.
%    \begin{macrocode}
 x_height#:=vstretch*carol_height#; 
%    \end{macrocode}
%
% \DescribeVariable{thickfudge}
%    The reciprocal of the font height in nib widths. Normal font height
% is 6 nib widths and the bold font height is 4.5 nib widths.
%    \begin{macrocode}
%<base7|base10|base17> thickfudge:=1/6;     
%<base7b|base10b|base17b> thickfudge:=1/4.5;     
%    \end{macrocode}
%
% \DescribeVariable{thinfudge}
%    The nib sharpness.
%    \begin{macrocode}
thinfudge:=1/6;      
%    \end{macrocode}
%
% \DescribeVariable{thick}
%    The maximum nib width. 
% That is, the width of the thickest line that can be penned.
%    \begin{macrocode}
thick#:=stemstretch*thickfudge*x_height#;    % max pen breadth
%    \end{macrocode}
%
% \DescribeVariable{thin}
%    The nib thinness. 
% That is, the width of the thinnest line that can be penned.
%    \begin{macrocode}
thin#:=thinfudge*thick#;  
%    \end{macrocode}
%
% \DescribeVariable{pangle}
%  The pen angle (in degrees).
%    \begin{macrocode} 
pangle:=45;                  % pen angle
%    \end{macrocode}
%
% \DescribeVariable{u}
%    The unit width. The CMR values are used.
%    \begin{macrocode}
%<base7> u#:=15.5/36pt#;
%<base7b> u#:=17.9/36pt#;  
%<base10> u#:=20/36pt#;  
%<base10b> u#:=23/36pt#;  
%<base17> u#:=32.5/36pt#;  
%<base17b> u#:=37/36pt#;  
%    \end{macrocode}
%
% \DescribeVariable{width_adj}
% \DescribeVariable{serif_fit}
% \DescribeVariable{cap_serif_fit}
% \DescribeVariable{letter_fit}
%    The majority of these parameters and values are constant.
% They are for fine adjustements of characters. The CM values are used.
%    \begin{macrocode}
 width_adj#:=0pt#;         % width adjustment for certain characters
 serif_fit#:=0pt#;         % extra sidebar near lowercase serifs
%<base7|base7b> cap_serif_fit#:=3.5/36pt#;  % extra sidebar near uppercase serifs
%<base10|base10b> cap_serif_fit#:=5/36pt#;  % extra sidebar near uppercase serifs
%<base17|base17b> cap_serif_fit#:=8/36pt#;  % extra sidebar near uppercase serifs
%<base7|base7b> letter_fit#:=0pt#;        % extra space added to all sidebars
%<base10|base10b> letter_fit#:=0pt#;        % extra space added to all sidebars
%<base17|base17b> letter_fit#:=-0.1pt#;        % extra space added to all sidebars

%    \end{macrocode}
%
%
% \DescribeVariable{asc_height} 
%    The height of lower case ascenders.
%    \begin{macrocode}
  asc_height#:=6/5x_height#;
%    \end{macrocode}
%
% \DescribeVariable{desc_depth} 
%    The depth of lower case descenders (the tail of the Q).
%    \begin{macrocode}
  desc_depth#:=2/3thick#;
%    \end{macrocode}
%
% \DescribeVariable{fig_height} 
%    The height of numerals. Make these midway between normal and capital
% letters.
%    \begin{macrocode}
 fig_height#:=(0.5[1,caprat])*x_height#;
%    \end{macrocode}
%
% \DescribeVariable{fig_width} 
%    The width of numerals. All numerals are the same width.
%    \begin{macrocode}
 fig_width#:=hstretch*(3/5x_height#);
%    \end{macrocode}
%
% \DescribeVariable{cap_height} 
%    The height of capital letters.
%    \begin{macrocode}
 cap_height#:=caprat*asc_height#;        % height of caps
%    \end{macrocode}
%
% \DescribeVariable{body_height} 
%    The height of the tallest character.
%    \begin{macrocode}
 body_height#:=caprat*asc_height#;    
%    \end{macrocode}
%
% \DescribeVariable{body_depth} 
%    The depth of the lowest character.
%    \begin{macrocode}
 body_depth#:=caprat*desc_depth#;    
%    \end{macrocode}
%
% \DescribeVariable{half_height} 
%    Half the normal letter height.
%    \begin{macrocode}
 half_height#:=1/2x_height#;    % half the height of x height letters

%    \end{macrocode}
%
% \DescribeVariable{side}
%    We might be using a square for the character design.
%    \begin{macrocode}
side#:=x_height#;     
%    \end{macrocode}
%
%
% \DescribeVariable{rule_thickness}
%    Thickness of rules (in math symbols). The CMR values are used.
%    \begin{macrocode}
%<base7|base7b> rule_thickness#:=.34pt#; 
%<base10|base10b> rule_thickness#:=.4pt#; 
%<base17|base17b> rule_thickness#:=.6pt#; 

%    \end{macrocode}
%
% \DescribeVariable{jutfudge}
%    Controls the protrusion of serifs.
%    \begin{macrocode}
jutfudge:= 5/8; 
%    \end{macrocode}
%
% \DescribeVariable{jut} 
%    The lowercase serif protrusion.
%    \begin{macrocode}
jut#:=jutstretch*jutfudge*thick#; 
%    \end{macrocode}
%
% \DescribeVariable{hookjut} 
%    The horizontal protrusion of hook serifs.
%    \begin{macrocode}
hookjut#:=1/2jut#;
%    \end{macrocode}
%
% \DescribeVariable{footjut} 
%    Half the horizontal width of a foot stroke, assuming vertical ends.
%    \begin{macrocode}
footjut#:=2/20x_height#;
%    \end{macrocode}
%
% \DescribeVariable{extra_fhjut} 
%    Half the extra horizontal width of a foot or head stroke, taking sloped
% ends into account. Thus the maximum overall horizontal width is 
% |2(footjut+extra_fhjut)|.
%    \begin{macrocode}
extra_fhjut#:=1/3thick#;
%    \end{macrocode}
%
% \DescribeVariable{stem_offset} 
%    The x offset of the center of a left (stem + foot) from the left of the 
% character box.
%    \begin{macrocode}
stem_offset#:=8/12thick#+thin#;
%    \end{macrocode}
%
% \DescribeVariable{srad}
%    |srad| with |0 < srad < 1| controls the serif radius.
%    \begin{macrocode}
srad:=0.5;   
%    \end{macrocode}
%
% \DescribeVariable{wiggle}
%    Controls the amount of wiggle in a |squiggle|, with |0 <= wiggle <1|.
%    \begin{macrocode}
wiggle:= 0.05;
%    \end{macrocode}
%
% \DescribeVariable{bowoff}
%    Controls the amount of bow in a |bow|, with |0 <= bowoff <1|.
%    \begin{macrocode}
bowoff:= 0.05;
%    \end{macrocode}
%
% \DescribeVariable{o}
%    Amount of overshoot for curves (as in an `o'). Use CMR values.
%    \begin{macrocode} 
%<base7|base7b> o#:=5/36pt#;   
%<base10|base10b> o#:=8/36pt#;   
%<base17|base17b> o#:=10/36pt#;   

%    \end{macrocode}
%
% \DescribeVariable{thick_tltobr}
%    A vector of the slope of thick diagonals from the top left point to the
% bottom right point. Use as:
% |ztl-zbr = whatever*thick_tltobr|.
%    \begin{macrocode} 
pair thick_tltobr;
thick_tltobr:=(-2,5);

%    \end{macrocode}
%
% \DescribeVariable{slant}
% |slant| is the amount the font slopes to the right. 
%    \begin{macrocode}
slant:=0;      % tilt ratio $(\Delta x/\Delta y)$

%    \end{macrocode}
% 
% \DescribeVariable{monospace} 
%    We are generating a variable-width font.
%    \begin{macrocode}
monospace:=false;    % should all characters have the same width?

%    \end{macrocode}
%
% \DescribeVariable{dot_size}
%    The size of (punctuation) dots.
%    \begin{macrocode}
dot_size#:=5thin#;   % size of dots
%    \end{macrocode}
%
% \DescribeVariable{comma_width}
%    The horizontal width of a comma.
%    \begin{macrocode}
comma_width#:=2.5dot_size#; 

%    \end{macrocode}
%
%    Finally, call the driver file for the font.
%    \begin{macrocode} 
generate rusttitle      %% switch to the driver file

%    \end{macrocode}
%
%    The end of this code section.
%    \begin{macrocode}
%</base7|base10|base17|base7b|base10b|base17b>
%    \end{macrocode}
%
%
% \subsection{The driver file}
%
%    The driver file sets up the font parameters and also contains
% the macros used in the definition of the characters.
%
%    \begin{macrocode}
%<*mfd>
% This is RUSTTITLE.MF for the Roman Rustic font.
% This makes a short font.
% This file also contains special macros universally needed in the 
% font but either not in the cmbase.mf file or modifications of these.
%
%    \end{macrocode}
%
% \begin{routine}{makebox}
%    This is a new version of the \Lpack{cmbase.mf} |makebox| macro.
% Essentially, as we are only generating a single case font, we don't
% need all the horizontals for the CMR characters.
%
%    \begin{macrocode}
% A new version of cmbase.mf makebox macro
def makebox(text rule) =
  for y=0,half_height,cap_height,body_height,x_height,-desc_depth,-body_depth:
    rule((l,y)t_,(r,y)t_); endfor  % horizontals
  for x=l,r: rule((x,-body_depth)t_,(x,body_height)t_); endfor % verticals
  for x=u*(1+floor(1/u)) step u until r-1:
    rule((x,-body_depth)t_,(x,body_height)t_); endfor         % more verticals
  if charic<>0:
    rule((r+charic*pt,h.o_),(r+charic*pt,.5h.o_)); fi  % italic correction
enddef;

%    \end{macrocode}
% \end{routine}
%
% \begin{routine}{proofpcbb}
%    The macro |proofpcbb(S, xoff, yoff)| draws crosses at the corners of the
% square, and at the bottom and top midpoints, where the point |zS| is at
% position |(xoff,yoff)| in the square. 
%    \begin{macrocode}

% A macro to delineate a letter's bounding square in proof mode.
% Point z$ is at (xoff,yoff) in the bounding square.
def proofpcbb(suffix $)(expr xoff, yoff) =
if proofing > 0:
  x$bl = x$ - xoff; y$bl = y$ - yoff;         % Bottom Left Hand Corner
  z$br = (x$bl+side, y$bl);                   % BRHC
  z$tr = (x$bl+side, y$bl+side);              % TRHC
  z$tl = (x$bl, y$bl+side);                   % TLHC
  z$bc = 1/2[z$bl,z$br]; z$tc = 1/2[z$tl,z$tr];  % horizontal mid points
  pair hc, vc;
  hc = (2u,0); vc = (0,2u);
  pickup pensquare scaled 2;
  for s = z$bl, z$br, z$tr, z$tl, z$bc, z$tc:
    draw s-hc--s+hc;
    draw s-vc--s+vc;
  endfor
fi
enddef;

%    \end{macrocode}
% \end{routine}
%
%
%
% \begin{routine}{rustic_font_setup}
%    This macro contains the set up for the fonts.
%    \begin{macrocode}
 
def rustic_font_setup= 
  define_pixels(side,half_height,carol_height,comma_width);
  define_pixels(u,dot_size,letter_size,cap_height,width_adj,serif_fit,
                cap_serif_fit,jut,cap_jut);
  define_pixels(hookjut,footjut,stem_offset,extra_fhjut,fig_width);
  define_whole_pixels(letter_fit);
  define_whole_vertical_pixels(body_height,asc_height,cap_height,x_height,
                               desc_depth,body_depth,fig_height);
  define_corrected_pixels(o);
  define_blacker_pixels(thick,thin);
  let adjust_fit = normal_adjust_fit;
%    \end{macrocode}
% \DescribeVariable{stylus}
%  The pen for drawing the thinnest lines is |stylus|.
%    \begin{macrocode}
  pickup pencircle scaled thin;
  stylus:=savepen;
%    \end{macrocode}
% \DescribeVariable{quill}
%  The normal calligraphic pen is |quill|.
%    \begin{macrocode}
  pickup pencircle xscaled thick yscaled thin rotated pangle;
  quill:=savepen;
%    \end{macrocode}
% \DescribeVariable{bowpen}
%  The pen for drawing a bow is |bowpen|.
%    \begin{macrocode}
  pickup pencircle xscaled thick yscaled thin rotated ((1/2(pangle+90)));
  bowpen:=savepen;
  numeric pcshiftx#; % x value of LHS of box wrt origin of char box
  define_pixels(pcshiftx);
  numeric sqxside#;   % x size of the square
  numeric sqyside#;   % y size of the square
  define_pixels(sqxside,sqyside);

enddef; % end of rustic_font_setup

%    \end{macrocode}
% \end{routine}
%
%
% \DescribeVariable{ductus}
% A boolean to control drawing the ductus (TRUE) or the full letter (FALSE).
%    \begin{macrocode}
boolean ductus;
ductus:=false;
%    \end{macrocode}
%
% \DescribeVariable{pdir}
% The vector corresponding to the pen angle.
%    \begin{macrocode}
pair pdir; pdir=dir(pangle);
%    \end{macrocode}
%
% \DescribeVariable{ppdir}
% The vector that is perpendicular (anticlockwise) to the pen angle.
%    \begin{macrocode}
pair ppdir; ppdir=dir(pangle+90);
%    \end{macrocode}
%
% \begin{routine}{draw_rusticstem}
% A macro to draw a rustic thin vertical stem.
% |draw_rusticstem(1,2)| draws the stem with top at |z1| and bottom at |z2|.
%    \begin{macrocode}

def draw_rusticstem(suffix $, $$)=
  numeric alpha, beta; alpha:=0.5thick;
  beta:=60;      % the ending angle
  pair rustic[];
  xpart rustic1=x$$; ypart rustic1=y$$+alpha*sind(beta);
  z$a=1/2[z$,rustic1];
  z$b=2/3[z$,rustic1];
  z$c=5/6[z$,rustic1];
  z$d=rustic1;
  z$ar=z$a shifted (alpha*up); z$al=z$a shifted (alpha*down);

  rustic2:=z$b shifted (alpha*right);
  z$br=rustic2 rotatedaround(z$b,1/4[90,beta]); z$bl=z$br rotatedaround(z$b,180);

  rustic2:=z$c shifted (alpha*right);
  z$cr=rustic2 rotatedaround(z$c,1/2[90,beta]); z$cl=z$cr rotatedaround(z$c,180);

  rustic2:=z$d shifted (alpha*right);
%%  z$dr=rustic2 rotatedaround(z$d,60); z$dl=z$dr rotatedaround(z$d,180);
  rustic3=rustic2 rotatedaround(z$d,beta);
  x$dr=xpart rustic3; y$dr=ypart rustic3 - thin;
  z$dl=z$dr rotatedaround(z$d,180);
  pen oldpen; oldpen:=currentpen;
  pickup stylus;
  draw z$--z$al;
%%  filldraw z$ar..z$br..z$cr..z$dr--z$dl..z$cl..z$bl..z$al--cycle;
  filldraw z$ar..z$br..z$cr..z$dr--z$dl..z$bl..z$al--cycle;
  pickup oldpen;
%%  labels($a,$al,$ar,$b,$bl,$br,$c,$cl,$cr,$dl,$dr);
  labels($a,$b,$c,$d,$dl,$dr);
enddef; % end of draw_rusticstem

%    \end{macrocode}
% \end{routine}
%
% \begin{routine}{draw_hserif}
% A macro to draw a hooked serif. |draw_hserif(1,2,f)| uses the |stylus| pen to
% draw a serif of extent |hookjut| at |z1| to the left of the line from |z2| 
% to |z1|. The
% radius is controlled by |f|, where $0<f<1$. The `stem' from |z1| to |z2|
% starts at the point |z1a|.
%    \begin{macrocode}

def draw_hserif(suffix $, $$)(expr fr)=
  pen oldpen; oldpen:=currentpen;
  numeric beta, ajut, bjut;
%    \end{macrocode}
% |fr| and |beta| control the serif radius.
%    \begin{macrocode}
  beta=1-fr;
  ajut=fr*jut; bjut=beta*jut;
%    \end{macrocode}
% 
% 
% 
% 
% 
%    \begin{macrocode}
  pair rustic;
  numeric tn, ang; ang:=70;
  tn:=(sind ang)/(cosd ang);
  x$g=x$-hookjut; y$g=y$-(hookjut*tn); % the tip
  x$f=x$; y$f=y$g;                     % on the stem
  pickup stylus;
%%  filldraw z$g{dir(ang)}..{down}z$f--z$--cycle;
  draw z$g--z$;
  pickup oldpen;
  labels($f,$g);
enddef; % end of draw_hserif

%    \end{macrocode}
% \end{routine}
%
% \begin{routine}{draw_footstem}
% A macro to draw a rustic stem and its foot. 
% |draw_footstem(1,2)| uses the current pen to draw
% the rustic stem with its top at |z1| and a foot at the point |z2|, 
% where |z2| is at the bottom of the foot.
% The width of the foot is twice the (global) value of |footjut|.
%    \begin{macrocode}

def draw_footstem(suffix $, $$)=
  draw_rusticstem($, $$);
  x$$fl=x$dl; bot y$$fl= y$$;
  x$$fr=x$$fl+2footjut; y$$fr=y$$fl;
  draw_squiggle($$fl,$$fr,wiggle,srad);
  labels($$fl,$$fr);
enddef; % end of draw_footstem

%    \end{macrocode}
% \end{routine}
%
% \begin{routine}{draw_foot}
% A macro to draw a foot. |draw_foot(1)| uses the current pen to draw
% a foot at the point |z1|, where |z1| is at the bottom center of the foot.
% The width of the foot is twice the (global) value of |footjut|.
%    \begin{macrocode}

def draw_foot(suffix $)=
  x$fl=x$-footjut; bot y$fl= y$;
  x$fr-x$=x$-x$fl; y$fr=y$fl;
  draw_squiggle($fl,$fr,wiggle,srad);
  labels($fl,$fr);
enddef; % end of draw_foot

%    \end{macocode}
% \end{routine}
%
% \begin{routine}{draw_head}
% A macro to draw a head squiggle. |draw_head(1)| uses the current pen to draw
% a head at the point |z1|, where |z1| is at the top center of the head.
% The width of the head is twice the (global) value of |footjut|.
%    \begin{macrocode}

def draw_head(suffix $)=
  x$hl=x$-footjut; top y$hl= y$;
  x$hr-x$=x$-x$hl; y$hr=y$hl;
  draw_squiggle($hl,$hr,wiggle,srad);
  labels($hl,$hr);
enddef; % end of draw_head

%    \end{macrocode}
% \end{routine}
%
% \begin{routine}{draw_bow}
% A macro to draw a bow. |draw_bow(1,2,f)| uses the |bowpen| to draw a bowed
% line from |z1| to |z2|. The midpoint of the bowed line is offest by an
% amount |f*l| perpendicular to the line joining |z1| to |z2|, where |l|
% is the distance between |z1| and |z2|.
%    \begin{macrocode}

def draw_bow(suffix $, $$)(expr fr)=
  pen oldpen; oldpen:=currentpen;
  z$bm=1/2[z$,z$$];
  l:=length(z$-z$$);
  bdir:=angle(z$-z$$);
%%  z$bo=z$bm shifted (fr*l*dir(bdir+90));
  z$bo=z$bm shifted (fr*l*dir(bdir-90));
  pickup bowpen;
  draw z$..z$bo..z$$;
  pickup oldpen;
  labels($bm,$bo);
enddef;  % end of draw_bow

%    \end{macrocode}
% \end{routine}
%
% \begin{routine}{draw_squiggle}
% A macro to draw a squiggle. |draw_squiggle(1,2,w,f)| draws a squiggle from
% |z1| to |z2| (z1 assumed to be at the left of z2), with |w| amount of
% wiggle and |f| controlling the serif radius.
%    \begin{macrocode}

def draw_squiggle(suffix $, $$)(expr w, fr)=
  z$si=1/4[z$,z$$];  z$sii=3/4[z$,z$$];
  l:=length(z$-z$$);
  bdir:=angle(z$-z$$);
%%  z$sio=z$si shifted (w*l*dir(bdir+90));
%%  z$siio=z$sii shifted (w*l*dir(bdir-90));
  z$sio=z$si shifted (w*l*dir(bdir-90));
  z$siio=z$sii shifted (w*l*dir(bdir+90));
  draw z$..z$sio..z$siio..z$$;    % the squiggle
%%  draw_serif($,$$,fr);
%%  draw_serif($$,$,fr);
  labels($sio,$siio);
enddef;  % end of draw_squiggle

%    \end{macrocode}
% \end{routine}
%
% \begin{routine}{draw_serif}
% A macro to draw a serif. |draw_serif(1,2,f)| uses the current pen to
% draw a serif of extent |jut| at |z1| to the left of the line from |z2| 
% to |z1|. The
% radius is controlled by |f|, where $0<f<1$. The `stem' from |z1| to |z2|
% starts at the point |z1a|.
%    \begin{macrocode}

def draw_serif(suffix $, $$)(expr fr)=
  numeric beta, ajut, bjut;
%    \end{macrocode}
% |fr| and |beta| control the serif radius.
%    \begin{macrocode}
  beta=1-fr;
  ajut=fr*jut; bjut=beta*jut;
%    \end{macrocode}
% The points |$c|, |$b| and |$a| define the serif. For a top (bottom) serif
% the leftmost (rightmost) point is 
% |$c| and |$a| is on the stem. |$a| is moved from |$| towards |$$| by the 
% distance |ajut|, and the other points are calculated from this. If the stem
% is not vertical, then we have to rotate |$b| to match.
%    \begin{macrocode}
  pair vec[]; vec1=unitvector(z$$-z$);
  vec2=ajut*vec1;
  z$a=z$ shifted vec2;
  z$b=z$a rotatedaround (z$,-90);
  boolean top_serif;
  if y$>y$$:
    top_serif := true;
    z$c=z$b shifted (-bjut*pdir);
  else:
    if y$=y$$:
      if x$<x$$:
        top_serif := true;
        z$c=z$b shifted (-bjut*pdir);
      else:
        top_serif := false;
        z$c=z$b shifted (bjut*pdir);
      fi
    else:
      top_serif:=false;
      z$c=z$b shifted (bjut*pdir);
    fi
  fi
%    \end{macrocode}
% Draw the parts, and finish off.
%    \begin{macrocode}
  if top_serif:
    draw z$c--z$b{pdir}..{z$$-z$}z$a;    % the serif
  else:
    draw z$c--z$b{-pdir}..{z$$-z$}z$a;
  fi
  labels($a,$b,$c);
enddef; % end of draw_serif

%    \end{macrocode}
% \end{routine}
%
% \begin{routine}{draw_istem}
%  A macro to draw the shape of an i. |draw_istem(1,2,f)| draws a stem whose
% top is at |z1| and bottom at |z2|. It draws with the current pen and the serifs
% are of extent |jut|. The serif radius is controlled by |f|, with $0<f<1$.
%    \begin{macrocode}

def draw_istem(suffix $, $$)(expr fr)=
  draw_serif($, $$, fr);  % top serif
  draw_serif($$, $, fr);  % bottom serif
  draw z$a--z$$a;  % the stem
enddef; % end of draw_istem

%    \end{macrocode}
% \end{routine}
%
% \begin{routine}{draw_tlserif}
%  A macro to draw a top left serif. |draw_tlserif(1,2,f)| draws a serif on top
% of a stem whose absolute 
% top is at |z1| and bottom at |z2|. It draws with the current pen and the serif
% is of extent |jut|. It also draws the stem to the point |z2|. The serif
% radius is controlled by |f|, with $0<f<1$.
%    \begin{macrocode}

def draw_tlserif(suffix $, $$)(expr fr)=
  draw_serif($, $$, fr);  % the serif
  draw z$a--z$$;                       % the stem
enddef; % end of draw_tlserif

%    \end{macrocode}
% \end{routine}
%
%
%
%
% \begin{routine}{draw_pdot}
%  A macro to draw a period dot.
% |draw_pdot(1,sz)| uses |stylus| to draw a dot centered at |z1| of overall size |sz|.
%    \begin{macrocode}

def draw_pdot(suffix $)(expr sz)=
%    \end{macrocode}
% Draw a square rotated 45 degrees.
%    \begin{macrocode}
  z$s = z$ shifted (0.5sz*left);
  z$t = z$ shifted (0.5sz*up);
  z$u = z$s shifted (sz*right);
  z$v = z$t shifted (sz*down);
  pickup stylus;
  filldraw z$s--z$t--z$u--z$v--cycle;
  labels($s,$t,$u,$v);
enddef; % end of draw_pdot

%    \end{macrocode}
% \end{routine}
%
% \begin{routine}{draw_pcomma}
%  A macro to draw a period comma.
% |draw_pcomma(1,t,w)| draws a comma placed at |z1| (the top), line width is |t|
% and bottom of line is |w| horizontally from |z1|. The pen is |stylus|.
%    \begin{macrocode}

def draw_pcomma(suffix $)(expr sz, ar)=
%    \end{macrocode}
% Draw a line of at |z$| of width |sz| sloping back and down at 45 degrees
% for a total width of |ar|.
%    \begin{macrocode}
  z$s = z$ shifted (0.5sz*left);
  z$t = z$s shifted (sz*right);
  x$v=x$s-ar; y$v=y$s-ar;
  z$u = z$v shifted (sz*right);
  pickup stylus;
  filldraw z$s--z$t--z$u--z$v--cycle;
  labels($s,$t,$u,$v);
enddef; % end of draw_pcomma

%    \end{macrocode}
% \end{routine}
%
% \begin{routine}{draw_plq}
%  A macro to draw a period left quote.
% |draw_pcomma(1,t,w)| draws a left quote placed at |z1| (the top), line width is |t|
% and bottom of line is |w| horizontally from |z1|.
%    \begin{macrocode}

def draw_plq(suffix $)(expr sz, ar)=
%    \end{macrocode}
% Draw a line of at |z$| of width |sz| sloping forward and down at 45 degrees
% for a total width of |ar|.
%    \begin{macrocode}
  z$s = z$ shifted (0.5sz*left);
  z$t = z$s shifted (sz*right);
  x$v=x$s+ar; y$v=y$s-ar;
  z$u = z$v shifted (sz*right);
  pickup stylus;
  filldraw z$s--z$t--z$u--z$v--cycle;
  labels($s,$t,$u,$v);
enddef; % end of draw_plq

%    \end{macrocode}
% \end{routine}
%
%
%    Set the mode and font setup.
%    \begin{macrocode}
 
font_coding_scheme:="ASCII minuscules and punctuation";
mode_setup; 
rustic_font_setup;

%    \end{macrocode}
%
%
%    We generate letters, together with some punctuation and
% analphabetics.
%    \begin{macrocode}
 
input rustl;        % normal letters
input rustu;        % capital letters
input rustdig;      % arabic digits
input rustpunct;    % punctuation
input rustlig;      % the several dashes
%%%%input rustp;        % ampersand and ?mark

%    \end{macrocode}
%
% Finally, do the uppercase ligtable.
%
%    \begin{macrocode}
 
font_slant slant; font_x_height x_height#;
if monospace: font_normal_space 9u#; % no stretching or shrinking
 font_quad 18u#;
 font_extra_space 9u#;
else: font_normal_space 6u#+2letter_fit#;
 font_normal_stretch 3u#; font_normal_shrink 2u#;
 font_quad 18u#+4letter_fit#;
 font_extra_space 2u#;
 k#:=-.5u#; kk#:=-1.5u#; kkk#:=-2u#; % three degrees of kerning
%% ligtable "P": "T": "Y": "A" kern kk#;
%% ligtable "F": "V": "W": "A" kern if serifs: kkk# else: kk#fi,
%%  "K": "X": "O" kern k#, "C" kern k#, "G" kern k#, "Q" kern k#;
%% ligtable "O": "D": "X" kern k#, "W" kern k#, "A" kern k#,
%%   "V" kern k#, "Y" kern k#;
%% ligtable "A": if serifs: "R": fi
%%  "C" kern k#, "O" kern k#, "G" kern k#, "U" kern k#, "Q" kern k#,
%%  "L": "T" kern kk#, "Y" kern kkk#, "V" kern kk#, "W" kern kkk#;
%%ligtable "I": "I" kern -k#; % Richard III
fi

bye.

%</mfd>
%    \end{macrocode}
%
% \subsection{The code for normal letters}
%
%    The following code is for the normal alphabetic characters.
%
%    \begin{macrocode}
%<*min>
% RUSTL.MF Program file for normal rustic letters
%


%    \end{macrocode}
%
%
% \begin{routine}{a}
% The letter `a'.
%    \begin{macrocode}
cmchar "Rustic letter a";
beginchar("a", 2u#+hstretch*(4/5x_height#+jut#), x_height#, 0);
adjust_fit(0,0);
pcshiftx := u;
pickup quill;
% right leg
z1-z2=whatever*thick_tltobr;
top y1=h;
rt x2=w-pcshiftx-jut; bot y2=0;
% for the left leg
z3=9/10[z2,z1];
x4=pcshiftx+footjut+extra_fhjut; bot y4=0;
% code for foot
x7=x4; y7=0;
%%ductus:=true;
if ductus:
  pickup stylus;
fi
draw_istem(1,2,srad);           % right leg
draw_bow(3,4,bowoff);           % left leg
draw_foot(7);                   % left leg foot
ductus:=false;
penlabels(1,2,3,4,5,6,100);
endchar; % end "a"

%    \end{macrocode}
% \end{routine}
%
% \begin{routine}{b}
% The letter `b'. This is a tall letter
%    \begin{macrocode}
cmchar "Rustic letter b";
beginchar("b", 2u#+hstretch*(7/10x_height#), 11/10x_height#, 0);
adjust_fit(0,0);
pcshiftx := u+hookjut+thin;
pickup quill;
% stem
x1=x2=pcshiftx;
y1=h; y2=0;
% bottom bowl
rt x3=x1; bot y3=thin;
rt x5=w-u; y5 = 1/6h;
x4=0.4[x2,x5]; y4=0;
lft x6=x2; y6=2/3h; 
% top bowl
z7=z6;
x8=0.25[x4,x5]; top y8=y1;
x9=0.5[x1,x8]; bot y9=h;
z10=z1;
%%ductus:=true;
if ductus:
  pickup stylus;
fi
draw_rusticstem(1,2);                                 % the stem
draw z3..z4{right}..z5..{dir(pangle+90)}z6;         % bottom bowl
draw z7{pdir}..z8{up}..z9{left}..z10;                 % top bowl
ductus:=false;
penlabels(1,2,3,4,5,6,7,8,9,10,100);
endchar; % end "b"

%    \end{macrocode}
% \end{routine}
%
%
% \begin{routine}{c}
% The letter `c'.
%    \begin{macrocode}
cmchar "Rustic letter c";
beginchar("c", 2u#+hstretch*(3/5x_height#), x_height#, 0);
adjust_fit(0,0);
pcshiftx := u;
pickup quill;
lft x1=pcshiftx; rt x3=pcshiftx+sqxside; y1=y3=h/2;
x2=x4=w/2; top y2=h; bot y4=0;
rt x5= rt x6=w-pcshiftx;
top y5=1/5h; 
top y6=4/5h;
%%ductus:=true;
if ductus:
  pickup stylus;
fi
draw z5{-pdir}..{left}z4..z1..z2..z6;
ductus:=false;
penlabels(1,2,3,4,5,6,100);
endchar; % end "c"

%    \end{macrocode}
% \end{routine}
%
% \begin{routine}{d}
% The letter `d'.
%    \begin{macrocode}
cmchar "Rustic letter d";
beginchar("d", 2u#+hstretch*(3/5x_height#), x_height#, 0);
adjust_fit(0,0);
pcshiftx := u;
pickup quill;
x1=x2=pcshiftx+hookjut+thin;
top y1=h; y2=0;
% bowl
rt x3=x1; bot y3=thin;
rt x5=w-u; y5 = 2/5h; 
x4=0.4[x2,x5]; y4=0;
x6=x1; top y6=h; 
%%ductus:=true;
if ductus:
  pickup stylus;
fi
draw_rusticstem(1,2);            % stem
draw z3..z4{right}..z5{up}..z6;  % bowl
ductus:=false;
penlabels(0,1,2,3,4,5,6,100);
endchar; % end "d"

%    \end{macrocode}
% \end{routine}
%
% \begin{routine}{e}
% The letter `e'.
%    \begin{macrocode}
cmchar "Rustic letter e";
beginchar("e", 2u#+hstretch*(4/10x_height#), x_height#, 0);
adjust_fit(0,0);
pcshiftx := u;
pickup quill;
% stem
x1=x2=pcshiftx+stem_offset;
top y1=h; y2=0;
% top bar
rt x3=x1; x4=w-u;
top y3=top y4=h;
% middle bar
x5=x3; x6=x4;
y5=y6=3/5h;
% foot
x7=x3; x8=x4;
bot y7= bot y8=0;
%%ductus:=true;
if ductus:
  pickup stylus;
fi
draw_rusticstem(1,2);            % stem
draw_squiggle(3,4,wiggle,srad);  % top bar
draw_squiggle(5,6,wiggle,srad);  % middle bar
draw_squiggle(7,8,wiggle,srad);  % bottom bar
ductus:=false;
penlabels(1,2,3,4,5,6,7,8,100);
endchar; % end "e"

%    \end{macrocode}
% \end{routine}
%
% \begin{routine}{f}
% The letter `f'. This is a tall letter
%    \begin{macrocode}
cmchar "Rustic letter f";
beginchar("f", 2u#+hstretch*(4/10x_height#), 6/5x_height#, 0);
adjust_fit(0,0);
pcshiftx := u +stem_offset;
pickup quill;
% stem
x1=x2=pcshiftx;
top y1=h; y2=0;
% top bar
x3=x1; x4=w-u;
top y3=top y4=h;
% middle bar
x5=x3; x6=x4;
y5=y6=2/3h;
% foot
x7=x3; x8=x4;
bot y7= bot y8=0;
%%ductus:=true;
if ductus:
  pickup stylus;
fi
draw_footstem(1,2);            % stem
draw_squiggle(3,4,wiggle,srad);  % top bar
draw_squiggle(5,6,wiggle,srad);  % middle bar
ductus:=false;
penlabels(1,2,3,4,5,6,7,8,100);
endchar; % end "f"

%    \end{macrocode}
% \end{routine}
%
%
% \begin{routine}{g}
% The letter `g'. Like a `c' but bottom curls inwards.
%    \begin{macrocode}
cmchar "Rustic letter g";
beginchar("g", 2u#+hstretch*(3/5x_height#), x_height#, 0);
adjust_fit(0,0);
pcshiftx := u;
pickup quill;
lft x1=pcshiftx; rt x3=pcshiftx+sqxside; y1=y3=h/2;
x2=x4=w/2; top y2=h; bot y4=0;
rt x5= rt x6=w-pcshiftx;
top y5=1/5h; 
top y6=4/5h;
lft x7=w/2; y7=h/2;
%%ductus:=true;
if ductus:
  pickup stylus;
fi
draw z7..z5{down}..{left}z4..z1..z2..z6;  % the bowl
ductus:=false;
penlabels(1,2,3,4,5,6,7,100);
endchar; % end "g"

%    \end{macrocode}
% \end{routine}
%
% \begin{routine}{h}
% The letter `h'.
%    \begin{macrocode}
cmchar "Rustic letter h";
beginchar("h", 2u#+hstretch*(4/5x_height#), x_height#, 0);
adjust_fit(0,0);
pcshiftx := u;
y100 = sqyside/2; x100 = pcshiftx+y100;  % square center
pickup quill;
% left stem and foot
x1=x2=pcshiftx+stem_offset;
y1=h; y2=0;
% bar
y5=y6=7/10h;
lft x5=pcshiftx; rt x6=w-pcshiftx;
% right stem
rt x7=w-pcshiftx; top y7=h;
x8=x7; bot y8=0;
x9=x8-2thin; y9=0.5[y8,y7];
%%ductus:=true;
if ductus:
  pickup stylus;
fi
draw_footstem(1,2);              % left stem
draw_hserif(1,2,srad);           % its serif
draw_squiggle(5,6,wiggle,srad);  % the bar
draw z7..z9..z8;                 % right stem
ductus:=false;
penlabels(1,1',2,2',3,4,5,6,7,7',8,8',99,99',100);
endchar; % end "h"

%    \end{macrocode}
% \end{routine}
%
% \begin{routine}{i}
% The letter i.
%    \begin{macrocode}
cmchar "Rustic letter i";
beginchar("i", 2u#+hstretch*(2footjut#+hookjut#), x_height#, 0);
adjust_fit(0,0);
pcshiftx := u;
pickup quill;
x1=x2=pcshiftx+stem_offset;
y1=h; y2=0;
%%ductus:=true;
if ductus:
  pickup stylus;
fi
draw_footstem(1,2);               % stem and foot
draw_hserif(1,2,srad);            % serif
ductus:=false;
penlabels(1,2,3,4,100);
endchar; % end "i"

%    \end{macrocode}
% \end{routine}
%
% \begin{routine}{j}
% The letter `j'. This letter was not in the script, it being a later invention.
%    \begin{macrocode}
cmchar "Rustic letter j";
beginchar("j", 4u#+hstretch*(3/5x_height#), x_height#, 0);
adjust_fit(0,0);
pcshiftx := 3u;
pickup quill;
% stem
x1=x2=pcshiftx+2/3(w-4u);
y1=h; y2=0;
% top bar
rt x4=w-u; top y4=h;
x1-x3 = x4-x1; y3=y4;
% bowl
rt x5 = lft x3;
top y5=2/5h;
x7=x2; bot y7=0;
%%ductus:=true;
if ductus:
  pickup stylus;
fi
draw_rusticstem(1,2);              % stem
draw_squiggle(3,4,wiggle,srad);    % top bar
draw z7{left}..{pdir}z5;           % bowl
ductus:=false;
penlabels(1,2,3,4,5,6,7,100);
endchar; % end "j"

%    \end{macrocode}
% \end{routine}
%
% \begin{routine}{k}
% The letter `k'.
%    \begin{macrocode}
cmchar "Rustic letter k";
beginchar("k", 2u#+hstretch*(x_height#), x_height#, 0);
adjust_fit(0,0);
pcshiftx := u;
pickup quill;
% the stem and its foot
x1=x2=pcshiftx+stem_offset;
y1=h; y2=0;
lft x13=pcshiftx; bot y13=0;
x2-x13=x14-x2; y14=y13;
% upper diagonal and its top
x3=x1; y3=h/2;
x4=x1+y3; top y4=h;
x5=x4-jut; x6-x4=x4-x5;
top y5=top y6=h;
x24=x4; y24=h;
% bottom leg
lft x7=x3; bot y7=y3;
rt x9=w-pcshiftx; bot y9=0;
x8=0.5[x7,x9]; y8=0.7[y7,y9];
%%ductus:=true;
if ductus:
  pickup stylus;
fi
draw_footstem(1,2);             % the stem
draw_hserif(1,2,srad);            % its serif
draw z3--z4;                      % upper diagonal
draw_head(24);
draw z7..z8..z9;                  % bottom leg
ductus:=false;
penlabels(1,2,3,4,5,6,7,8,9,100);
endchar; % end "k"

%    \end{macrocode}
% \end{routine}
%
% \begin{routine}{l}
% The letter `l'. This is a tall letter.
%    \begin{macrocode}
cmchar "Rustic letter l";
beginchar("l", 2u#+hstretch*(1/2x_height#), 6/5x_height#, 0);
adjust_fit(0,0);
pcshiftx := u+hookjut;
pickup quill;
x1=x2=pcshiftx+thin+hookjut;
y1=h; y2=0;
rt x3=x1; rt x4=w-u;
bot y3= bot y4=0;
%%ductus:=true;
if ductus:
  pickup stylus;
fi
draw_rusticstem(1,2);            % stem
draw_hserif(1,2,srad);           % serif
draw_squiggle(3,4,wiggle,srad);  % foot
ductus:=false;
penlabels(1,2,3,4,100);
endchar; % end "l"

%    \end{macrocode}
% \end{routine}
%
% \begin{routine}{m}
% The letter `m'. Like 2 As stuck together.
%    \begin{macrocode}
cmchar "Rustic letter m";
beginchar("m", 2u#+hstretch*(6/5x_height#+jut#), x_height#, 0);
adjust_fit(0,0);
pcshiftx := u;
pickup quill;
% right a thick diag
z11-z12=whatever*thick_tltobr;
rt x12=w-pcshiftx-jut; bot y12=0;
top y11=h;
% left a thin diag
pickup bowpen;
x4=pcshiftx+footjut+extra_fhjut; bot y4=thin;
z3=9/10[z2,z1];
pickup quill;
x7=x4; y7=0;   % for the foot
% the two thin diags are parallel, and just shifted horizontally
z13=9/10[z12,z11];
z3-z4=z13-z14; y4=y14;
% similarly for the two thick diags
z1-z2=z11-z12; % y2=y12;
x12-x2=1/3w;
%%ductus:=true;
if ductus:
  pickup stylus;
fi
% draw left a
draw_istem(1,2,srad);           % right leg
draw_bow(3,4,bowoff);           % left leg
draw_foot(7);                   % left leg foot
% draw right a
draw_istem(11,12,srad);
draw_bow(13,14,bowoff);
ductus:=false;
penlabels(1,2,3,4,5,6,11,12,13,14,100);
endchar; % end "m"

%    \end{macrocode}
% \end{routine}
%
% \begin{routine}{n}
% The letter `n'. 
%    \begin{macrocode}
cmchar "Rustic letter n";
beginchar("n", 2u#+hstretch*(9/10x_height#), x_height#, 0);
adjust_fit(0,0);
pcshiftx := u;
pickup quill;
% left stem and foot
x1=x2=pcshiftx+stem_offset;
y1=h; y2=0;
% right stem
x5=x6=w-u-hookjut-thin;
y5=y1; y6=y2;
% diagonal
x7=x1; y7=h;
x9=x6; bot y9=0;
x8=0.5[x7,x9]; y8=0.6[y7,y9];
%%ductus:=true;
if ductus:
  pickup stylus;
fi
draw_footstem(1,2);            % left stem
draw_rusticstem(5,6);            % right stem
draw_hserif(5,6,srad);           % its serif
draw z7..z8..z9;                 % diagonal
draw_serif(9,8,srad);            % its bottom serif
ductus:=false;
penlabels(1,2,5,6,7,8,9,100);
endchar; % end "n"

%    \end{macrocode}
% \end{routine}
%
% \begin{routine}{o}
% The letter `o'.
%    \begin{macrocode}
cmchar "Rustic letter o";
beginchar("o", 2u#+hstretch*(3/5x_height#), x_height#, 0);
adjust_fit(0,0);
pcshiftx := u;
pickup quill;
lft x1=pcshiftx; rt x3=w-pcshiftx; y1=y3=h/2;
x2=x4=w/2; top y2=h+o; bot y4=-o;
%%ductus:=true;
if ductus:
  pickup stylus;
fi
draw z1..z2..z3..z4..cycle;
ductus:=false;
penlabels(1,2,3,4,100);
endchar; % end "o"

%    \end{macrocode}
% \end{routine}
%
% \begin{routine}{p}
% The letter `p'.
%    \begin{macrocode}
cmchar "Rustic letter p";
beginchar("p", 2u#+hstretch*(5/10x_height#), x_height#, 0);
adjust_fit(0,0);
pcshiftx := u;
pickup quill;
% stem and foot
x1=x2=pcshiftx+stem_offset;
top y1=h; y2=0;
% bowl
lft x7=x2; y7=1/2h;  % y7=2/3h;
rt x8=w-pcshiftx; top y8=y1;
z10=z1;
x9=0.5[x1,x8]; y9=h;
%%ductus:=true;
if ductus:
  pickup stylus;
fi
draw_footstem(1,2);                    % stem
draw z7{pdir}..z8{up}..z9{left}..z10;    % top bowl
ductus:=false;
penlabels(1,2,7,8,9,10,100);
endchar; % end "p"

%    \end{macrocode}
% \end{routine}
%
% \begin{routine}{q}
% The letter `q'. 
%    \begin{macrocode}
cmchar "Rustic letter q";
beginchar("q", 2u#+hstretch*(3/5x_height#), x_height#, 2/3thick#);
adjust_fit(0,0);
pcshiftx := u;
pickup quill;
% the O shape
lft x1=pcshiftx; rt x3=w-pcshiftx; y1=y3=h/2;
x2=x4=w/2; top y2=h+o; bot y4=-o;
% the tail
path p;
p = z4{right}..{up}z3;
z5 = directionpoint pdir of p;
x7= x4+w/2; top y7=0;
x6=0.5[x5,x7]; y6=y7;
%%ductus:=true;
if ductus:
  pickup stylus;
fi
draw z1..z2..z3..z4..cycle;  % O shape
draw z5..z6..z7;             % the tail
ductus:=false;
penlabels(0,1,2,3,4,5,6,6',7,100);
endchar; % end "q"

%    \end{macrocode}
% \end{routine}
%
% \begin{routine}{r}
% The letter `r'.
%    \begin{macrocode}
cmchar "Rustic letter r";
beginchar("r", 2u#+hstretch*(4/5x_height#), x_height#, 0);
adjust_fit(0,0);
pcshiftx := u;
numeric alpha, beta;
alpha:=thin*(xpart pdir); beta:=thin*(ypart pdir);
pickup quill;
% stem
x1=x2=pcshiftx+stem_offset;
top y1=h; y2=0;
% bowl
lft x7=x2; bot y7=1/2h;
rt x8=5/8w; top y8=y1;
z10=z1;
x9=0.5[x1,x8]; y9=h;
% leg
z11=z7 shifted (thin*pdir);
rt x13=w-pcshiftx; bot y13=0;
x12=0.5[x11,x13]; y12=0.7[y11,y13];
% points defining the leg profile
x11l=lft x11 + alpha; y11l=bot y11 + beta;
x11r= rt x11 - alpha; y11r=top y11 - beta;
x12l=0.6[lft x12, x12]; y12l=0.6[bot y12, y12];
x12r=0.6[rt x12, x12];  y12r=0.6[top y12, y12];
x14=w-pcshiftx-thin; y14=thin;
%%ductus:=true;
if ductus:
  pickup stylus;
fi
draw_footstem(1,2);                       % stem and foot
draw z7{pdir}..z8{up}..z9{left}..z10;     % bowl
pickup stylus;                            % drawing the leg
filldraw z11l..z12l..{right}z14 & z14{left}..z12r..z11r--cycle;
ductus:=false;
penlabels(1,2,7,8,9,10,11,12,14,100);
endchar; % end "r"

%    \end{macrocode}
% \end{routine}
%
% \begin{routine}{s}
% The letter `s'.
%    \begin{macrocode}
cmchar "Rustic letter s";
beginchar("s", 2u#+hstretch*(3/5x_height#), x_height#, 0);
adjust_fit(0,0);
pcshiftx := u;
pickup quill;
% for the foot
lft x7=pcshiftx+extra_fhjut; bot y7=0;
% the S curve
rt x1=w-pcshiftx; top y1=h-thin;
lft x3=pcshiftx+extra_fhjut; y3=0.75h;
x4=0.5[x1,x3]; y4=0.5[y3,y5];
x2=x4; top y2=h;
x5=x1; y5=0.3h;
x6=0.75w; bot y6=0;
%%ductus:=true;
if ductus:
  pickup stylus;
fi
draw z1..z2{left}..z3{down}..z4..z5{down}..{-pdir}z6;  % S curve
draw_squiggle(7,6,wiggle,srad);                       % base
ductus:=false;
penlabels(1,2,3,4,5,6,7,100);
endchar; % end "s"

%    \end{macrocode}
% \end{routine}
%
% \begin{routine}{t}
% The letter `t'.
%    \begin{macrocode}
cmchar "Rustic letter t";
beginchar("t", 2u#+hstretch*(2/5x_height#), x_height#, 0);
adjust_fit(0,0);
pcshiftx := u;
pickup quill;
x1=x2=w/2;
y1=h; y2=0;
lft x3=pcshiftx; top y3=h;
rt x4=w-pcshiftx; y4=y3;
rt x5=x2; bot y5= bot y6=0;
rt x6=w-pcshiftx;
%%ductus:=true;
if ductus:
  pickup stylus;
fi
draw_footstem(1,2);            % the stem
draw_squiggle(3,4,wiggle,srad);  % the top crossbar
ductus:=false;
penlabels(1,2,3,4,5,6,100);
endchar; % end "t"

%    \end{macrocode}
% \end{routine}
%
% \begin{routine}{u}
% The letter `u'. This is a modern version as the font did not have a U.
%    \begin{macrocode}
cmchar "Rustic letter u";
beginchar("u", 2u#+hstretch*(4/5x_height#), x_height#, 0);
adjust_fit(0,0);
pcshiftx := u;
pickup quill;
% the right stem
x5=x6=w-pcshiftx-hookjut-thin; y5=h; y6=-thin;
% left stem and bowl
lft x1= lft x2=pcshiftx+jut;
top y1=h; y2=2thick;
x3=0.5[x1,x5]; bot y3=0;
rt x4=x5; bot y4=0.25thick;
%%ductus:=true;
if ductus:
  pickup stylus;
fi
draw_rusticstem(5,6);                       % right stem
draw_hserif(5,6,srad);                      % its serif
draw_tlserif(1,2,srad);                     % left stem and serif
draw z2{down}..z3{right}..{pdir}z4;         % the bowl
ductus:=false;
penlabels(1,2,3,4,5,6,7,8,9,11,16,100);
endchar; % end "u"

%    \end{macrocode}
% \end{routine}
%
% \begin{routine}{v}
% The letter `v'. 
% if you prefer).
%    \begin{macrocode}
cmchar "Rustic letter v";
beginchar("v", 2u#+hstretch*(x_height#), x_height#, 0);
adjust_fit(0,0);
pcshiftx := u;
pickup quill;
% left side
z1-z2=whatever*thick_tltobr;
lft x1=pcshiftx+jut; top y1=h;
bot y2=0;
% right side
pickup bowpen;
x3=w-pcshiftx-footjut-extra_fhjut; top y3=h;
pickup quill;
x4=lft x2; y4=0;
x5=x3; y5=h;
%%ductus:=true;
if ductus:
  pickup stylus;
fi
draw_istem(1,2,srad);      % left part
draw_bow(3,4,bowoff); % right part
draw_head(5);         % and its head
ductus:=false;
penlabels(1,2,3,4,5,100);
endchar; % end "v"

%    \end{macrocode}
% \end{routine}
%
% \begin{routine}{w}
% The letter `w'. This is a modern version as the font did not have a W.
%    \begin{macrocode}
cmchar "Rustic letter w";
beginchar("w", 2u#+hstretch*(8/5x_height#), x_height#, 0);
adjust_fit(0,0);
pcshiftx := u;
pickup quill;
% left thick diagonal
z1-z2=whatever*thick_tltobr;
lft x1=pcshiftx+jut; top y1=h; bot y2=0;
% right thin diagonal
pickup bowpen;
x13=w-pcshiftx-footjut-extra_fhjut; top y13=h-thin;
x23=x13; y23=h;  % for the head
bot y14=thin;
pickup quill;
% right thick diagonal
z11-z12=z1-z2;
x12-x2=1/3w; y12=y2;
% left thin diagonal
z3-z4=z13-z14;
x4=x2;
x14=x12;
y3=y13;

%%ductus:=true;
if ductus:
  pickup stylus;
fi
draw_istem(1,2,srad);       % left and right thick diags
draw_istem(11,12,srad);
draw_bow(3,4,bowoff);       % left thin diag
draw_bow(13,14,bowoff);     % right thin diag
draw_head(23);              % its head
ductus:=false;
penlabels(1,2,3,4,11,12,13,14,23,100);
endchar; % end "w"

%    \end{macrocode}
% \end{routine}
%
% \begin{routine}{x}
% The letter `x'.
%    \begin{macrocode}
cmchar "Rustic letter x";
beginchar("x", 2u#+hstretch*(9/10x_height#), x_height#, 0);
adjust_fit(0,0);
pcshiftx := u;
pickup quill;
% thick diagonal
lft x1=pcshiftx+jut; top y1=h;
rt x2= w-pcshiftx-jut; bot y2=0;
% thin diagonal
pickup bowpen;
x3=w-pcshiftx-footjut-extra_fhjut-thin; top y3=h;
x4=pcshiftx+footjut+extra_fhjut+thin; bot y4=0;
x7=x3; y7=h;
x8=x4; y8=0;
pickup quill;
%%ductus:=true;
if ductus:
  pickup stylus;
fi
draw_istem(1,2,srad);            % thick diagonal
draw_bow(3,4,bowoff);            % thin diagonal
draw_head(7);                    % its head
draw_foot(8);                    % its foot
ductus:=false;
penlabels(1,2,3,4,5,6,7,8,100);
endchar; % end "x"

%    \end{macrocode}
% \end{routine}
%
% \begin{routine}{y}
% The letter `y'. 
%    \begin{macrocode}
cmchar "Rustic letter y";
beginchar("y", 2u#+hstretch*(4/5x_height#), x_height#, 0);
adjust_fit(0,0);
pcshiftx := u;
pickup quill;
% left side
lft x1=pcshiftx+jut; top y1=h;
x2=1/2w; bot y2=1/3h;
% right side
pickup bowpen;
x3=w-pcshiftx-footjut-extra_fhjut; top y3=h;
lft x4=x2; bot y4=y2;
pickup quill;
x5=x3; y5=h;
% leg
x7=x2; y7=y2;
x8=x7; y8=0;
pickup bowpen;
x10=lft x8; x11=rt x8;
pickup stylus;
bot y10= bot y11=thin;
pickup quill;
%% ductus:=true;
if ductus:
  pickup stylus;
fi
draw_istem(1,2,srad);     % left stroke
draw_bow(3,4,bowoff);     % right stroke
draw_head(5);             % its head
draw_foot(8);             % the foot
pickup stylus;            % the leg
filldraw z7--z10--z11--cycle;
ductus:=false;
penlabels(1,2,3,4,5,7,8,9,100);
endchar; % end "y"

%    \end{macrocode}
% \end{routine}
%
%
% \begin{routine}{z}
% The letter `z'.
%    \begin{macrocode}
cmchar "Rustic letter z";
beginchar("z", 2u#+hstretch*(4/5x_height#), x_height#, 0);
adjust_fit(0,0);
pcshiftx := u;
pickup quill;
% bottom stroke
lft x4=pcshiftx; bot y4=0;
rt x6=w-pcshiftx; y6=y4;
% top stroke
x1=x4; top y1=h;
rt x3 =7/8w-pcshiftx; y3=y1;
%%ductus:=true;
if ductus:
  pickup stylus;
fi
draw_squiggle(1,3,1/2wiggle,srad);    % top stroke
draw_squiggle(4,6,1/2wiggle,srad);    % bottom stroke
% the diagonal
pickup stylus;
draw z4--z3;
ductus:=false;
penlabels(1,2,3,4,5,6,100);
endchar; % end "z"

%    \end{macrocode}
% \end{routine}
%
%
%    The end of this file
%    \begin{macrocode} 
%</min>
%    \end{macrocode}
%
%
% \subsection{The code for capitals}
%
%    Roman Rustic `capitals' were drawn with the same pen as for the normal
% text, the letters were merely drawn slightly larger.
%
%    \begin{macrocode}
%<*maj>
% RUSTU.MF Program file for Roman Rustic `capitals'
%

%    \end{macrocode}
%
%
% \begin{routine}{A}
% The letter `A'.
%    \begin{macrocode}
cmchar "Rustic letter A";
beginchar("A", 2u#+caprat*hstretch*(4/5x_height#+jut#), caprat*x_height#, 0);
adjust_fit(0,0);
pcshiftx := u;
pickup quill;
% right leg
z1-z2=whatever*thick_tltobr;
top y1=h;
rt x2=w-pcshiftx-jut; bot y2=0;
% for the left leg
z3=9/10[z2,z1];
x4=pcshiftx+footjut+extra_fhjut; bot y4=0;
% code for foot
x7=x4; y7=0;
%%ductus:=true;
if ductus:
  pickup stylus;
fi
draw_istem(1,2,srad);           % right leg
draw_bow(3,4,bowoff);           % left leg
draw_foot(7);                   % left leg foot
ductus:=false;
penlabels(1,2,3,4,5,6,100);
endchar; % end "A"

%    \end{macrocode}
% \end{routine}
%
% \begin{routine}{B}
% The letter `B'. This is a tall letter
%    \begin{macrocode}
cmchar "Rustic letter B";
beginchar("B", 2u#+caprat*hstretch*(7/10x_height#), caprat*(11/10x_height#), 0);
adjust_fit(0,0);
pcshiftx := u+hookjut+thin;
pickup quill;
% stem
x1=x2=pcshiftx;
y1=h; y2=0;
% bottom bowl
rt x3=x1; bot y3=thin;
rt x5=w-u; y5 = 1/6h; 
x4=0.4[x2,x5]; y4=0;
lft x6=x2; y6=2/3h; 
% top bowl
z7=z6;
x8=0.25[x4,x5]; top y8=y1;
x9=0.5[x1,x8]; bot y9=h;
z10=z1;
%%ductus:=true;
if ductus:
  pickup stylus;
fi
draw_rusticstem(1,2);                                 % the stem
draw z3..z4{right}..z5..{dir(pangle+90)}z6;         % bottom bowl
draw z7{pdir}..z8{up}..z9{left}..z10;                 % top bowl
ductus:=false;
penlabels(1,2,3,4,5,6,7,8,9,10,100);
endchar; % end "B"

%    \end{macrocode}
% \end{routine}
%
%
% \begin{routine}{C}
% The letter `C'.
%    \begin{macrocode}
cmchar "Rustic letter C";
beginchar("C", 2u#+caprat*hstretch*(3/5x_height#), caprat*x_height#, 0);
adjust_fit(0,0);
pcshiftx := u;
pickup quill;
lft x1=pcshiftx; rt x3=pcshiftx+sqxside; y1=y3=h/2;
x2=x4=w/2; top y2=h; bot y4=0;
rt x5= rt x6=w-pcshiftx;
top y5=1/5h; 
top y6=4/5h;
%%ductus:=true;
if ductus:
  pickup stylus;
fi
draw z5{-pdir}..{left}z4..z1..z2..z6;
ductus:=false;
penlabels(1,2,3,4,5,6,100);
endchar; % end "C"

%    \end{macrocode}
% \end{routine}
%
% \begin{routine}{D}
% The letter `D'.
%    \begin{macrocode}
cmchar "Rustic letter D";
beginchar("D", 2u#+caprat*hstretch*(3/5x_height#), caprat*x_height#, 0);
adjust_fit(0,0);
pcshiftx := u;
pickup quill;
x1=x2=pcshiftx+hookjut+thin;
top y1=h; y2=0;
% bowl
rt x3=x1; bot y3=thin;
rt x5=w-u; y5 = 2/5h; 
x4=0.4[x2,x5]; y4=0;
x6=x1; top y6=h; 
%%ductus:=true;
if ductus:
  pickup stylus;
fi
draw_rusticstem(1,2);            % stem
draw z3..z4{right}..z5{up}..z6;  % bowl
ductus:=false;
penlabels(0,1,2,3,4,5,6,100);
endchar; % end "D"

%    \end{macrocode}
% \end{routine}
%
% \begin{routine}{E}
% The letter `E'.
%    \begin{macrocode}
cmchar "Rustic letter E";
beginchar("E", 2u#+caprat*hstretch*(4/10x_height#), caprat*x_height#, 0);
adjust_fit(0,0);
pcshiftx := u;
pickup quill;
% stem
x1=x2=pcshiftx+stem_offset;
top y1=h; y2=0;
% top bar
rt x3=x1; x4=w-u;
top y3=top y4=h;
% middle bar
x5=x3; x6=x4;
y5=y6=3/5h;
% foot
x7=x3; x8=x4;
bot y7= bot y8=0;
%%ductus:=true;
if ductus:
  pickup stylus;
fi
draw_rusticstem(1,2);            % stem
draw_squiggle(3,4,wiggle,srad);  % top bar
draw_squiggle(5,6,wiggle,srad);  % middle bar
draw_squiggle(7,8,wiggle,srad);  % bottom bar
ductus:=false;
penlabels(1,2,3,4,5,6,7,8,100);
endchar; % end "E"

%    \end{macrocode}
% \end{routine}
%
% \begin{routine}{F}
% The letter `F'. This is a tall letter
%    \begin{macrocode}
cmchar "Rustic letter F";
beginchar("F", 2u#+caprat*hstretch*(4/10x_height#), caprat*(6/5x_height#), 0);
adjust_fit(0,0);
pcshiftx := u +stem_offset;
pickup quill;
% stem
x1=x2=pcshiftx;
top y1=h; y2=0;
% top bar
x3=x1; x4=w-u;
top y3=top y4=h;
% middle bar
x5=x3; x6=x4;
y5=y6=2/3h;
% foot
x7=x3; x8=x4;
bot y7= bot y8=0;
%%ductus:=true;
if ductus:
  pickup stylus;
fi
draw_footstem(1,2);            % stem
draw_squiggle(3,4,wiggle,srad);  % top bar
draw_squiggle(5,6,wiggle,srad);  % middle bar
ductus:=false;
penlabels(1,2,3,4,5,6,7,8,100);
endchar; % end "F"

%    \end{macrocode}
% \end{routine}
%
%
% \begin{routine}{G}
% The letter `G'. Like a `C' but bottom curls inwards.
%    \begin{macrocode}
cmchar "Rustic letter G";
beginchar("G", 2u#+caprat*hstretch*(3/5x_height#), caprat*x_height#, 0);
adjust_fit(0,0);
pcshiftx := u;
pickup quill;
lft x1=pcshiftx; rt x3=pcshiftx+sqxside; y1=y3=h/2;
x2=x4=w/2; top y2=h; bot y4=0;
rt x5= rt x6=w-pcshiftx;
top y5=1/5h; 
top y6=4/5h;
lft x7=w/2; y7=h/2;
%%ductus:=true;
if ductus:
  pickup stylus;
fi
draw z7..z5{down}..{left}z4..z1..z2..z6;  % the bowl
ductus:=false;
penlabels(1,2,3,4,5,6,7,100);
endchar; % end "G"

%    \end{macrocode}
% \end{routine}
%
% \begin{routine}{H}
% The letter `H'.
%    \begin{macrocode}
cmchar "Rustic letter H";
beginchar("H", 2u#+caprat*hstretch*(4/5x_height#), caprat*x_height#, 0);
adjust_fit(0,0);
pcshiftx := u;
y100 = sqyside/2; x100 = pcshiftx+y100;  % square center
pickup quill;
% left stem and foot
x1=x2=pcshiftx+stem_offset;
y1=h; y2=0;
% bar
y5=y6=7/10h;
lft x5=pcshiftx; rt x6=w-pcshiftx;
% right stem
rt x7=w-pcshiftx; top y7=h;
x8=x7; bot y8=0;
x9=x8-2thin; y9=0.5[y8,y7];
%%ductus:=true;
if ductus:
  pickup stylus;
fi
draw_footstem(1,2);              % left stem
draw_hserif(1,2,srad);           % its serif
draw_squiggle(5,6,wiggle,srad);  % the bar
draw z7..z9..z8;                 % right stem
ductus:=false;
penlabels(1,1',2,2',3,4,5,6,7,7',8,8',99,99',100);
endchar; % end "H"

%    \end{macrocode}
% \end{routine}
%
% \begin{routine}{I}
% The letter I.
%    \begin{macrocode}
cmchar "Rustic letter I";
beginchar("I", 2u#+caprat*hstretch*(2footjut#+hookjut#), caprat*x_height#, 0);
adjust_fit(0,0);
pcshiftx := u;
pickup quill;
x1=x2=pcshiftx+stem_offset;
y1=h; y2=0;
%%ductus:=true;
if ductus:
  pickup stylus;
fi
draw_footstem(1,2);               % stem and foot
draw_hserif(1,2,srad);            % serif
ductus:=false;
penlabels(1,2,3,4,100);
endchar; % end "I"

%    \end{macrocode}
% \end{routine}
%
% \begin{routine}{J}
% The letter `J'. This letter was not in the script, it being a later invention.
%    \begin{macrocode}
cmchar "Rustic letter J";
beginchar("J", 4u#+caprat*hstretch*(3/5x_height#), caprat*x_height#, 0);
adjust_fit(0,0);
pcshiftx := 3u;
pickup quill;
% stem
x1=x2=pcshiftx+2/3(w-4u);
y1=h; y2=0;
% top bar
rt x4=w-u; top y4=h;
x1-x3 = x4-x1; y3=y4;
% bowl
rt x5 = lft x3;
top y5=2/5h;
x7=x2; bot y7=0;
%%ductus:=true;
if ductus:
  pickup stylus;
fi
draw_rusticstem(1,2);              % stem
draw_squiggle(3,4,wiggle,srad);    % top bar
draw z7{left}..{pdir}z5;           % bowl
ductus:=false;
penlabels(1,2,3,4,5,6,7,100);
endchar; % end "J"

%    \end{macrocode}
% \end{routine}
%
% \begin{routine}{K}
% The letter `K'.
%    \begin{macrocode}
cmchar "Rustic letter K";
beginchar("K", 2u#+caprat*hstretch*(x_height#), caprat*x_height#, 0);
adjust_fit(0,0);
pcshiftx := u;
pickup quill;
% the stem and its foot
x1=x2=pcshiftx+stem_offset;
y1=h; y2=0;
lft x13=pcshiftx; bot y13=0;
x2-x13=x14-x2; y14=y13;
% upper diagonal and its top
x3=x1; y3=h/2;
x4=x1+y3; top y4=h;
x5=x4-jut; x6-x4=x4-x5;
top y5=top y6=h;
x24=x4; y24=h;
% bottom leg
lft x7=x3; bot y7=y3;
rt x9=w-pcshiftx; bot y9=0;
x8=0.5[x7,x9]; y8=0.7[y7,y9];
%%ductus:=true;
if ductus:
  pickup stylus;
fi
draw_footstem(1,2);             % the stem
draw_hserif(1,2,srad);            % its serif
draw z3--z4;                      % upper diagonal
draw_head(24);
draw z7..z8..z9;                  % bottom leg
ductus:=false;
penlabels(1,2,3,4,5,6,7,8,9,100);
endchar; % end "K"

%    \end{macrocode}
% \end{routine}
%
% \begin{routine}{L}
% The letter `L'. This is a tall letter.
%    \begin{macrocode}
cmchar "Rustic letter L";
beginchar("L", 2u#+caprat*hstretch*(1/2x_height#), caprat*(6/5x_height#), 0);
adjust_fit(0,0);
pcshiftx := u+hookjut;
pickup quill;
x1=x2=pcshiftx+thin+hookjut;
y1=h; y2=0;
rt x3=x1; rt x4=w-u;
bot y3= bot y4=0;
%%ductus:=true;
if ductus:
  pickup stylus;
fi
draw_rusticstem(1,2);            % stem
draw_hserif(1,2,srad);           % serif
draw_squiggle(3,4,wiggle,srad);  % foot
ductus:=false;
penlabels(1,2,3,4,100);
endchar; % end "L"

%    \end{macrocode}
% \end{routine}
%
% \begin{routine}{M}
% The letter `M'. Like 2 As stuck together.
%    \begin{macrocode}
cmchar "Rustic letter M";
beginchar("M", 2u#+caprat*hstretch*(6/5x_height#+jut#), caprat*x_height#, 0);
adjust_fit(0,0);
pcshiftx := u;
pickup quill;
% right a thick diag
z11-z12=whatever*thick_tltobr;
rt x12=w-pcshiftx-jut; bot y12=0;
top y11=h;
% left a thin diag
pickup bowpen;
x4=pcshiftx+footjut+extra_fhjut; bot y4=thin;
z3=9/10[z2,z1];
pickup quill;
x7=x4; y7=0;   % for the foot
% the two thin diags are parallel, and just shifted horizontally
z13=9/10[z12,z11];
z3-z4=z13-z14; y4=y14;
% similarly for the two thick diags
z1-z2=z11-z12; % y2=y12;
x12-x2=1/3w;
%%ductus:=true;
if ductus:
  pickup stylus;
fi
% draw left a
draw_istem(1,2,srad);           % right leg
draw_bow(3,4,bowoff);           % left leg
draw_foot(7);                   % left leg foot
% draw right a
draw_istem(11,12,srad);
draw_bow(13,14,bowoff);
ductus:=false;
penlabels(1,2,3,4,5,6,11,12,13,14,100);
endchar; % end "M"

%    \end{macrocode}
% \end{routine}
%
% \begin{routine}{N}
% The letter `N'. 
%    \begin{macrocode}
cmchar "Rustic letter N";
beginchar("N", 2u#+caprat*hstretch*(9/10x_height#), caprat*x_height#, 0);
adjust_fit(0,0);
pcshiftx := u;
pickup quill;
% left stem and foot
x1=x2=pcshiftx+stem_offset;
y1=h; y2=0;
% right stem
x5=x6=w-u-hookjut-thin;
y5=y1; y6=y2;
% diagonal
x7=x1; y7=h;
x9=x6; bot y9=0;
x8=0.5[x7,x9]; y8=0.6[y7,y9];
%%ductus:=true;
if ductus:
  pickup stylus;
fi
draw_footstem(1,2);            % left stem
draw_rusticstem(5,6);            % right stem
draw_hserif(5,6,srad);           % its serif
draw z7..z8..z9;                 % diagonal
draw_serif(9,8,srad);            % its bottom serif
ductus:=false;
penlabels(1,2,5,6,7,8,9,100);
endchar; % end "N"

%    \end{macrocode}
% \end{routine}
%
% \begin{routine}{O}
% The letter `O'.
%    \begin{macrocode}
cmchar "Rustic letter O";
beginchar("O", 2u#+caprat*hstretch*(3/5x_height#), caprat*x_height#, 0);
adjust_fit(0,0);
pcshiftx := u;
pickup quill;
lft x1=pcshiftx; rt x3=w-pcshiftx; y1=y3=h/2;
x2=x4=w/2; top y2=h+o; bot y4=-o;
%%ductus:=true;
if ductus:
  pickup stylus;
fi
draw z1..z2..z3..z4..cycle;
ductus:=false;
penlabels(1,2,3,4,100);
endchar; % end "O"

%    \end{macrocode}
% \end{routine}
%
% \begin{routine}{P}
% The letter `P'.
%    \begin{macrocode}
cmchar "Rustic letter P";
beginchar("P", 2u#+caprat*hstretch*(5/10x_height#), caprat*x_height#, 0);
adjust_fit(0,0);
pcshiftx := u;
pickup quill;
% stem and foot
x1=x2=pcshiftx+stem_offset;
top y1=h; y2=0;
% bowl
lft x7=x2; y7=1/2h;  % y7=2/3h;
rt x8=w-pcshiftx; top y8=y1;
z10=z1;
x9=0.5[x1,x8]; y9=h;
%%ductus:=true;
if ductus:
  pickup stylus;
fi
draw_footstem(1,2);                    % stem
draw z7{pdir}..z8{up}..z9{left}..z10;    % top bowl
ductus:=false;
penlabels(1,2,7,8,9,10,100);
endchar; % end "P"

%    \end{macrocode}
% \end{routine}
%
% \begin{routine}{Q}
% The letter `Q'. 
%    \begin{macrocode}
cmchar "Rustic letter Q";
beginchar("Q", 2u#+caprat*hstretch*(3/5x_height#), caprat*x_height#, 
                                                   caprat*(2/3thick#));
adjust_fit(0,0);
pcshiftx := u;
pickup quill;
% the O shape
lft x1=pcshiftx; rt x3=w-pcshiftx; y1=y3=h/2;
x2=x4=w/2; top y2=h+o; bot y4=-o;
% the tail
path p;
p = z4{right}..{up}z3;
z5 = directionpoint pdir of p;
x7= x4+w/2; top y7=0;
x6=0.5[x5,x7]; y6=y7;
%%ductus:=true;
if ductus:
  pickup stylus;
fi
draw z1..z2..z3..z4..cycle;  % O shape
draw z5..z6..z7;             % the tail
ductus:=false;
penlabels(0,1,2,3,4,5,6,6',7,100);
endchar; % end "Q"

%    \end{macrocode}
% \end{routine}
%
% \begin{routine}{R}
% The letter `R'.
%    \begin{macrocode}
cmchar "Rustic letter R";
beginchar("R", 2u#+caprat*hstretch*(4/5x_height#), caprat*x_height#, 0);
adjust_fit(0,0);
pcshiftx := u;
numeric alpha, beta;
alpha:=thin*(xpart pdir); beta:=thin*(ypart pdir);
pickup quill;
% stem
x1=x2=pcshiftx+stem_offset;
top y1=h; y2=0;
% bowl
lft x7=x2; bot y7=1/2h;
rt x8=5/8w; top y8=y1;
z10=z1;
x9=0.5[x1,x8]; y9=h;
% leg
z11=z7 shifted (thin*pdir);
rt x13=w-pcshiftx; bot y13=0;
x12=0.5[x11,x13]; y12=0.7[y11,y13];
% points defining the leg profile
x11l=lft x11 + alpha; y11l=bot y11 + beta;
x11r= rt x11 - alpha; y11r=top y11 - beta;
x12l=0.6[lft x12, x12]; y12l=0.6[bot y12, y12];
x12r=0.6[rt x12, x12];  y12r=0.6[top y12, y12];
x14=w-pcshiftx-thin; y14=thin;
%%ductus:=true;
if ductus:
  pickup stylus;
fi
draw_footstem(1,2);                       % stem and foot
draw z7{pdir}..z8{up}..z9{left}..z10;     % bowl
pickup stylus;                            % drawing the leg
filldraw z11l..z12l..{right}z14 & z14{left}..z12r..z11r--cycle;
ductus:=false;
penlabels(1,2,7,8,9,10,11,12,14,100);
endchar; % end "R"

%    \end{macrocode}
% \end{routine}
%
% \begin{routine}{S}
% The letter `S'.
%    \begin{macrocode}
cmchar "Rustic letter S";
beginchar("S", 2u#+caprat*hstretch*(3/5x_height#), caprat*x_height#, 0);
adjust_fit(0,0);
pcshiftx := u;
pickup quill;
% for the foot
lft x7=pcshiftx+extra_fhjut; bot y7=0;
% the S curve
rt x1=w-pcshiftx; top y1=h-thin;
lft x3=pcshiftx+extra_fhjut; y3=0.75h;
x4=0.5[x1,x3]; y4=0.5[y3,y5];
x2=x4; top y2=h;
x5=x1; y5=0.3h;
x6=0.75w; bot y6=0;
%%ductus:=true;
if ductus:
  pickup stylus;
fi
draw z1..z2{left}..z3{down}..z4..z5{down}..{-pdir}z6;  % S curve
draw_squiggle(7,6,wiggle,srad);                       % base
ductus:=false;
penlabels(1,2,3,4,5,6,7,100);
endchar; % end "S"

%    \end{macrocode}
% \end{routine}
%
% \begin{routine}{T}
% The letter `T'.
%    \begin{macrocode}
cmchar "Rustic letter T";
beginchar("T", 2u#+caprat*hstretch*(2/5x_height#), caprat*x_height#, 0);
adjust_fit(0,0);
pcshiftx := u;
pickup quill;
x1=x2=w/2;
y1=h; y2=0;
lft x3=pcshiftx; top y3=h;
rt x4=w-pcshiftx; y4=y3;
rt x5=x2; bot y5= bot y6=0;
rt x6=w-pcshiftx;
%%ductus:=true;
if ductus:
  pickup stylus;
fi
draw_footstem(1,2);            % the stem
draw_squiggle(3,4,wiggle,srad);  % the top crossbar
ductus:=false;
penlabels(1,2,3,4,5,6,100);
endchar; % end "T"

%    \end{macrocode}
% \end{routine}
%
% \begin{routine}{U}
% The letter `U'. This is a modern version as the font did not have a U.
%    \begin{macrocode}
cmchar "Rustic letter U";
beginchar("U", 2u#+caprat*hstretch*(4/5x_height#), caprat*x_height#, 0);
adjust_fit(0,0);
pcshiftx := u;
pickup quill;
% the right stem
x5=x6=w-pcshiftx-hookjut-thin; y5=h; y6=-thin;
% left stem and bowl
lft x1= lft x2=pcshiftx+jut;
top y1=h; y2=2thick;
x3=0.5[x1,x5]; bot y3=0;
rt x4=x5; bot y4=0.25thick;
%%ductus:=true;
if ductus:
  pickup stylus;
fi
draw_rusticstem(5,6);                       % right stem
draw_hserif(5,6,srad);                      % its serif
draw_tlserif(1,2,srad);                     % left stem and serif
draw z2{down}..z3{right}..{pdir}z4;         % the bowl
ductus:=false;
penlabels(1,2,3,4,5,6,7,8,9,11,16,100);
endchar; % end "U"

%    \end{macrocode}
% \end{routine}
%
% \begin{routine}{V}
% The letter `V'. 
% if you prefer).
%    \begin{macrocode}
cmchar "Rustic letter V";
beginchar("V", 2u#+caprat*hstretch*(x_height#), caprat*x_height#, 0);
adjust_fit(0,0);
pcshiftx := u;
pickup quill;
% left side
z1-z2=whatever*thick_tltobr;
lft x1=pcshiftx+jut; top y1=h;
bot y2=0;
% right side
pickup bowpen;
x3=w-pcshiftx-footjut-extra_fhjut; top y3=h;
pickup quill;
x4=lft x2; y4=0;
x5=x3; y5=h;
%%ductus:=true;
if ductus:
  pickup stylus;
fi
draw_istem(1,2,srad);      % left part
draw_bow(3,4,bowoff); % right part
draw_head(5);         % and its head
ductus:=false;
penlabels(1,2,3,4,5,100);
endchar; % end "V"

%    \end{macrocode}
% \end{routine}
%
% \begin{routine}{W}
% The letter `W'. This is a modern version as the font did not have a W.
%    \begin{macrocode}
cmchar "Rustic letter W";
beginchar("W", 2u#+caprat*hstretch*(8/5x_height#), caprat*x_height#, 0);
adjust_fit(0,0);
pcshiftx := u;
pickup quill;
% left thick diagonal
z1-z2=whatever*thick_tltobr;
lft x1=pcshiftx+jut; top y1=h; bot y2=0;
% right thin diagonal
pickup bowpen;
x13=w-pcshiftx-footjut-extra_fhjut; top y13=h-thin;
x23=x13; y23=h;  % for the head
bot y14=thin;
pickup quill;
% right thick diagonal
z11-z12=z1-z2;
x12-x2=1/3w; y12=y2;
% left thin diagonal
z3-z4=z13-z14;
x4=x2;
x14=x12;
y3=y13;
%%ductus:=true;
if ductus:
  pickup stylus;
fi
draw_istem(1,2,srad);       % left and right thick diags
draw_istem(11,12,srad);
draw_bow(3,4,bowoff);       % left thin diag
draw_bow(13,14,bowoff);     % right thin diag
draw_head(23);              % its head
ductus:=false;
penlabels(1,2,3,4,11,12,13,14,23,100);
endchar; % end "W"

%    \end{macrocode}
% \end{routine}
%
% \begin{routine}{X}
% The letter `X'.
%    \begin{macrocode}
cmchar "Rustic letter X";
beginchar("X", 2u#+caprat*hstretch*(9/10x_height#), caprat*x_height#, 0);
adjust_fit(0,0);
pcshiftx := u;
pickup quill;
% thick diagonal
lft x1=pcshiftx+jut; top y1=h;
rt x2= w-pcshiftx-jut; bot y2=0;
% thin diagonal
pickup bowpen;
x3=w-pcshiftx-footjut-extra_fhjut-thin; top y3=h;
x4=pcshiftx+footjut+extra_fhjut+thin; bot y4=0;
x7=x3; y7=h;
x8=x4; y8=0;
pickup quill;
%%ductus:=true;
if ductus:
  pickup stylus;
fi
draw_istem(1,2,srad);            % thick diagonal
draw_bow(3,4,bowoff);            % thin diagonal
draw_head(7);                    % its head
draw_foot(8);                    % its foot
ductus:=false;
penlabels(1,2,3,4,5,6,7,8,100);
endchar; % end "X"

%    \end{macrocode}
% \end{routine}
%
% \begin{routine}{Y}
% The letter `Y'. 
%    \begin{macrocode}
cmchar "Rustic letter Y";
beginchar("Y", 2u#+caprat*hstretch*(4/5x_height#), caprat*x_height#, 0);
adjust_fit(0,0);
pcshiftx := u;
pickup quill;
% left side
lft x1=pcshiftx+jut; top y1=h;
x2=1/2w; bot y2=1/3h;
% right side
pickup bowpen;
x3=w-pcshiftx-footjut-extra_fhjut; top y3=h;
lft x4=x2; bot y4=y2;
pickup quill;
x5=x3; y5=h;
% leg
x7=x2; y7=y2;
x8=x7; y8=0;
pickup bowpen;
x10=lft x8; x11=rt x8;
pickup stylus;
bot y10= bot y11=thin;
pickup quill;
%% ductus:=true;
if ductus:
  pickup stylus;
fi
draw_istem(1,2,srad);     % left stroke
draw_bow(3,4,bowoff);     % right stroke
draw_head(5);             % its head
draw_foot(8);             % the foot
pickup stylus;            % the leg
filldraw z7--z10--z11--cycle;
ductus:=false;
penlabels(1,2,3,4,5,7,8,9,100);
endchar; % end "Y"

%    \end{macrocode}
% \end{routine}
%
%
% \begin{routine}{Z}
% The letter `Z'.
%    \begin{macrocode}
cmchar "Rustic letter Z";
beginchar("Z", 2u#+caprat*hstretch*(4/5x_height#), caprat*x_height#, 0);
adjust_fit(0,0);
pcshiftx := u;
pickup quill;
% bottom stroke
lft x4=pcshiftx; bot y4=0;
rt x6=w-pcshiftx; y6=y4;
% top stroke
x1=x4; top y1=h;
rt x3 =7/8w-pcshiftx; y3=y1;
%%ductus:=true;
if ductus:
  pickup stylus;
fi
draw_squiggle(1,3,1/2wiggle,srad);    % top stroke
draw_squiggle(4,6,1/2wiggle,srad);    % bottom stroke
% the diagonal
pickup stylus;
draw z4--z3;
ductus:=false;
penlabels(1,2,3,4,5,6,100);
endchar; % end "Z"

%    \end{macrocode}
% \end{routine}
%
%
%
%
%    The end of this file
%    \begin{macrocode} 
%</maj>
%    \end{macrocode}
%
%
%
% \subsection{The punctuation file}
%
% The main punctuation marks are defined in this file.
%
%    \begin{macrocode}
%<*punct>
% RUSTPUNCT.MF This file contains punctuation marks

%    \end{macrocode}
%
%
% \begin{routine}{,}
%    The `,' character.
%    \begin{macrocode}

cmchar "Rustic Comma";
beginchar(",",2u#+hstretch*(comma_width#+dot_size#),x_height#,0);
adjust_fit(0,0);
pcshiftx:=u;
x1=w-pcshiftx-(0.5hstretch*dot_size); y1=h/2;
draw_pcomma(1,dot_size,comma_width);
penlabels(1); 
endchar; % end of ","

%    \end{macrocode}
% \end{routine}
%
% \begin{routine}{.}
%    The `.' character.
%    \begin{macrocode}

cmchar "Rustic Period";
beginchar(".",2u#+hstretch*(dot_size#),x_height#,0);
pcshiftx:=u;
x1=w/2; y1=h/2;
draw_pdot(1,dot_size);
penlabels(1); 
endchar; % end of "."

%    \end{macrocode}
% \end{routine}
%
%
% \begin{routine}{:}
%    The `:' character.
%    \begin{macrocode}

cmchar "Rustic Colon";
beginchar(":",2u#+hstretch*(dot_size#),x_height#,0);
pcshiftx:=u;
x1=x2=w/2; y1=2/3h; y2=1/3h;
draw_pdot(1,dot_size);
draw_pdot(2,dot_size);
penlabels(1,2); 
endchar; % end ":"

%    \end{macrocode}
% \end{routine}
%
% \begin{routine}{;}
%    The `;' character.
%    \begin{macrocode}

cmchar "Rustic Semicolon";
beginchar(";",2u#+hstretch*(comma_width#+dot_size#),x_height#,0);
adjust_fit(0,0);
pcshiftx:=u;
x1=x2=w-pcshiftx-(0.5hstretch*dot_size); y1=2/3h; y2=1/3h;
draw_pdot(1,dot_size);
draw_pcomma(2,dot_size,comma_width);
penlabels(1,2); 
endchar; % end of ";"

%    \end{macrocode}
% \end{routine}
%
% \begin{routine}{'}
%    The `'' character.
%    \begin{macrocode}

cmchar "Rustic Apostrophe";
beginchar("'",2u#+hstretch*(comma_width#+dot_size#),4/3x_height#,0);
adjust_fit(0,0);
pcshiftx:=u;
x1=w-pcshiftx-(0.5hstretch*dot_size); y1=h;
draw_pcomma(1,dot_size,comma_width);
penlabels(1); 
endchar;  % end of "'"

%    \end{macrocode}
% \end{routine}
%
% \begin{routine}{`}
%    The ``' character.
%    \begin{macrocode}

cmchar "Rustic Reverse apostrophe";
beginchar("`",2u#+hstretch*(comma_width#+dot_size#),4/3x_height#,0);
adjust_fit(0,0);
pcshiftx:=u;
x1=pcshiftx+(0.5hstretch*dot_size); y1=h;
draw_plq(1,dot_size,comma_width);
penlabels(1); 
endchar;  % end of "`"

%    \end{macrocode}
% \end{routine}
%
% The end of this file.
%    \begin{macrocode}
%</punct>
%    \end{macrocode}
%
%
% \subsection{The ligatures and dashes file}
%
%    The ligatured punctuation marks and dashes are defined in this file.
%    \begin{macrocode}
%<*lig>
% RUSTLIG.MF Roman Rustic ligatured punctation and dashes

%    \end{macrocode}
%  First the ligtables.
%    \begin{macrocode}

ligtable "`": "`"=:oct"134";
ligtable "'": "'"=:oct"042"; %% , "?" kern 2u#, "!" kern 2u#;

ligtable "-": "-"=:oct"173";
ligtable oct"173": "-"=:oct"174";

%    \end{macrocode}
%
% \begin{routine}{''}
%   The `''' ligature.
%    \begin{macrocode}

cmchar "Rustic closing quotes";
beginchar(oct"042",2u#+hstretch*(comma_width#+3dot_size#),4/3x_height#,0);
adjust_fit(0,0);
pcshiftx:=u;
x1=w-pcshiftx-(0.5hstretch*dot_size); y1=y2=h;
x2=x1-(2hstretch*dot_size);
draw_pcomma(1,dot_size,comma_width);  % right
draw_pcomma(2,dot_size,comma_width);  % left
penlabels(1,2); 
endchar; % end of "''"

%    \end{macrocode}
% \end{routine}
%
% \begin{routine}{``}
%   The ```' ligature.
%    \begin{macrocode}

cmchar "Rustic opening quotes";
beginchar(oct"134",2u#+hstretch*(comma_width#+3dot_size#),4/3x_height#,0);
adjust_fit(0,0);
pcshiftx:=u;
x1=pcshiftx+(0.5hstretch*dot_size); y1=y2=h;
x2=x1+(2hstretch*dot_size);
draw_plq(1,dot_size,comma_width); % left
draw_plq(2,dot_size,comma_width); % right
penlabels(1,2); 
endchar; % end of "``"

%    \end{macrocode}
% \end{routine}
%
% \begin{routine}{-}
%   The `-' character.
%    \begin{macrocode}

cmchar "Rustic hyphen";
beginchar("-",u#+hstretch*(0.8x_height#),x_height#,0);
adjust_fit(0,0);
pcshiftx:=u/2;
pickup quill;
lft x1 = pcshiftx; rt x2=w-pcshiftx; y1=y2=h/2;
draw z1--z2;  % bar
penlabels(1,2); 
endchar; % end of "-"

%    \end{macrocode}
% \end{routine}
%
% \begin{routine}{--}
%   The -- ligature. 
%    \begin{macrocode}

cmchar "Rustic en dash";
beginchar(oct"173",u#+hstretch*(x_height#),x_height#,0);
adjust_fit(0,0);
pcshiftx:=u/2;
pickup quill;
lft x1 = pcshiftx; rt x2=w-pcshiftx; y1=y2=h/2;
draw z1--z2;  % bar
penlabels(1,2); 
endchar; % end of "--"

%    \end{macrocode}
% \end{routine}
%
% \begin{routine}{---}
%   The `---' ligature. The em dash is twice the width of the en dash.
%    \begin{macrocode}

cmchar "Rustic em dash";
beginchar(oct"174",u#+hstretch*(2x_height#),x_height#,0);
adjust_fit(0,0);
pcshiftx:=u/2;
pickup quill;
lft x1 = pcshiftx; rt x2=w-pcshiftx; y1=y2=h/2;
draw z1--z2;  % bar
penlabels(1,2); 
endchar; % end of "---"

%    \end{macrocode}
% \end{routine}
%
%
% The end of this file.
%    \begin{macrocode}
%</lig>
%    \end{macrocode}
%
% \subsection{The digits file}
%
%    These are arabic digits in the style of the book-hand. They are a modern
% invention.
%
%    \begin{macrocode}
%<*dig>
% RUSTDIG.MF Program file for Roman Rustic `digits'
%

%    \end{macrocode}
% 
% \begin{routine}{1}
% The digit `1'.
%    \begin{macrocode}
cmchar "Rustic digit 1";
beginchar("1", 2u#+fig_width#, fig_height#, 0);
adjust_fit(0,0);
pcshiftx := u;
pickup quill;
x1=x2=w/2;
y1=h; y2=0;
%%ductus:=true;
if ductus:
  pickup stylus;
fi
draw_footstem(1,2);               % stem and foot
draw_hserif(1,2,srad);            % serif
ductus:=false;
penlabels(1,2,3,4,100);
endchar; % end "1"

%    \end{macrocode}
% \end{routine}
%
%
% \begin{routine}{2}
% The digit `2'. This is like a stemless B.
%    \begin{macrocode}
cmchar "Rustic digit 2";
beginchar("2", 2u#+fig_width#, fig_height#, 0);
adjust_fit(0,0);
pcshiftx := u;
pickup quill;
% stem
lft x1=pcshiftx; top y1=h-2thin;
x2=1/2w; top y2= h;
rt x3=w-pcshiftx; y3=3/4h;
% squiggle
x5=pcshiftx+extra_fhjut; bot y5=0;
x6=w-pcshiftx-extra_fhjut; y6=y5;
%%ductus:=true;
if ductus:
  pickup stylus;
fi
draw z1..z2{right}..z3{down}..{-pdir}z5;    % curve
draw_squiggle(5,6,wiggle,srad);             % foot
ductus:=false;
penlabels(1,2,3,4,5,6,100);
endchar; % end "2"

%    \end{macrocode}
% \end{routine}
%
%
% \begin{routine}{3}
% The digit `3'. This is like a stemless B.
%    \begin{macrocode}
cmchar "Rustic digit 3";
beginchar("3", 2u#+fig_width#, fig_height#, 0);
adjust_fit(0,0);
pcshiftx := u;
pickup quill;
% x values
lft x3=pcshiftx; x10=x3;
rt x5=w-pcshiftx; x8=x5-3thin;
x4=0.5[x3,x5];
x9=0.5[x10,x8];
lft x6=pcshiftx+extra_fhjut; x7=x6;
% y values
bot y3=thin; top y10=h-2thin;
y6=0.5[y3,y10]; y7=y6;
bot y4=-thin; top y9=h+thin;
y5=0.5[y3,y6];
y8=11/12[y7,y10];
%%ductus:=true;
if ductus:
  pickup stylus;
fi
%%draw_rusticstem(1,2);                                 % the stem
draw z3..z4{right}..z5{up}...{-pdir}z6;                 % bottom bowl
draw z7{pdir}..z8{up}..z9{left}..z10;                   % top bowl
ductus:=false;
penlabels(1,2,3,4,5,6,7,8,9,10,100);
endchar; % end "3"

%    \end{macrocode}
% \end{routine}
%
% \begin{routine}{4}
% The digit `4'.
%    \begin{macrocode}
cmchar "Rustic digit 4";
beginchar("4", 2u#+fig_width#, fig_height#, 0);
adjust_fit(0,0);
pcshiftx := u;
pickup quill;
% the stem
x1=x2=w-pcshiftx-footjut-extra_fhjut;
y1=h; y2=0;
% the crossbar
x3=pcshiftx+extra_fhjut; y3=0.33h;
x4=w-pcshiftx-extra_fhjut; y4=y3;
% the thin diagonal
x5=lft x3; y5=bot y3+thin;
%%ductus:=true;
if ductus:
  pickup stylus;
fi
draw_footstem(1,2);               % stem and foot
%%draw_hserif(1,2,srad);            % serif
draw_squiggle(3,4,wiggle,srad);   % squiggle
pickup stylus;
draw z5{pdir}..z1;                % thin diagonal
ductus:=false;
penlabels(1,2,3,4,100);
endchar; % end "4"

%    \end{macrocode}
% \end{routine}
%
%
%
% \begin{routine}{5}
% The digit `5'.
%    \begin{macrocode}
cmchar "Rustic digit 5";
beginchar("5", 2u#+fig_width#, fig_height#, 0);
adjust_fit(0,0);
pcshiftx := u;
pickup quill;
% for the foot
lft x7=pcshiftx+extra_fhjut; bot y7=0;
% for the top
rt x1=w-pcshiftx-extra_fhjut; top y1=h;
lft x2=pcshiftx+extra_fhjut; y2=y1;

% the S curve
x3=x2; y3=0.5h;
x4=0.5[x1,x3]; y4=0.5[y3,y5];
rt x5=w-pcshiftx; y5=0.3h;
x6=0.75w; bot y6=0;
%%ductus:=true;
if ductus:
  pickup stylus;
fi
draw_squiggle(2,1,wiggle,srad);            % top squiggle
draw z3{pdir}..z5{down}..{-pdir}z6;        % curve
draw_squiggle(7,6,wiggle,srad);            % base
pickup stylus;
draw z2--z3;                               % vertical
ductus:=false;
penlabels(1,2,3,4,5,6,7,100);
endchar; % end "5"

%    \end{macrocode}
% \end{routine}
%
% \begin{routine}{6}
% The digit `6'.
%    \begin{macrocode}
cmchar "Rustic digit 6";
beginchar("6", 2u#+fig_width#, fig_height#, 0);
adjust_fit(0,0);
pcshiftx := u;
pickup quill;
rt x1=w-pcshiftx; top y1=h-2thin;
x2=1/2w; top y2= h;
lft x3=pcshiftx; y3=1/2h;
x4=x2; bot y4=0;
rt x5=w-pcshiftx; y5=1/4h;
lft x6=rt x3; bot y6=y3;
%%ductus:=true;
if ductus:
  pickup stylus;
fi
draw z1..z2{left}..z3{down}..z4{right}..z5{up}..{-pdir}z6;
ductus:=false;
penlabels(1,2,3,4,5,6,100);
endchar; % end "6"

%    \end{macrocode}
% \end{routine}
%
%
% \begin{routine}{7}
% The digit `7'.
%    \begin{macrocode}
cmchar "Rustic digit 7";
beginchar("7", 2u#+fig_width#, fig_height#, 0);
adjust_fit(0,0);
pcshiftx := u;
pickup quill;
% top bar
x1=pcshiftx+extra_fhjut;
x2=w-pcshiftx-extra_fhjut;
top y1=top y2=h;
% bowed leg
rt x4=w-pcshiftx-3thin; bot y4=0;
x3=0.5[x2,x4]-2thin; y3=0.5[y2,y4];
%%ductus:=true;
if ductus:
  pickup stylus;
fi
draw_squiggle(1,2,wiggle,srad);
draw z2..z3..z4;
ductus:=false;
penlabels(1,2,3,4,5,6,100);
endchar; % end "7"

%    \end{macrocode}
% \end{routine}
%
% \begin{routine}{8}
% The digit `8'.
%    \begin{macrocode}
cmchar "Rustic digit 8";
beginchar("8", 2u#+fig_width#, fig_height#, 0);
adjust_fit(0,0);
pcshiftx := u;
pickup quill;
lft x1=pcshiftx-o; rt x3=w-pcshiftx; y1=y3=3/4h;
x2=x4=w/2; top y2=h+o; bot y4=-o;
x5=x2; y5=h/2;
x11=x1+o; x13=x3+o; y11=y13=1/4h;
%%ductus:=true;
if ductus:
  pickup stylus;
fi
draw z1..z2..z3..z5..z11..z4..z13..z5..cycle;
ductus:=false;
penlabels(1,2,3,4,5,11,13,100);
endchar; % end "8"

%    \end{macrocode}
% \end{routine}
%
% \begin{routine}{9}
% The digit `9'.
%    \begin{macrocode}
cmchar "Rustic digit 9";
beginchar("9", 2u#+fig_width#, fig_height#, 0);
adjust_fit(0,0);
pcshiftx := u;
pickup quill;
lft x1=pcshiftx; bot y1=2thin;
x2=1/2w; bot y2= 0;
rt x3=w-pcshiftx; y3=1/2h;
x4=x2; top y4=h;
lft x5=pcshiftx; y5=3/4h;
rt x6=lft x3; top y6=y3;
%%ductus:=true;
if ductus:
  pickup stylus;
fi
draw z1..z2{right}..z3{up}..z4{left}..z5{down}..{pdir}z6;
ductus:=false;
penlabels(1,2,3,4,5,6,100);
endchar; % end "9"

%    \end{macrocode}
% \end{routine}
%
% \begin{routine}{0}
% The digit `0'.
%    \begin{macrocode}
cmchar "Rustic digit 0";
beginchar("0", 2u#+fig_width#, fig_height#, 0);
adjust_fit(0,0);
pcshiftx := u;
pickup quill;
lft x1=pcshiftx; rt x3=w-pcshiftx; y1=y3=h/2;
x2=x4=w/2; top y2=h+o; bot y4=-o;
%%ductus:=true;
if ductus:
  pickup stylus;
fi
draw z1..z2..z3..z4..cycle;
ductus:=false;
penlabels(1,2,3,4,100);
endchar; % end "0"

%    \end{macrocode}
% \end{routine}
%
%
% The end of this file.
%    \begin{macrocode}
%</dig>
%    \end{macrocode}
%
%
% \section{The font definition files} \label{sec:fd}
%
%    The font comes in normal and bold weights only.
%
%    \begin{macrocode}
%<*fdot1>
\DeclareFontFamily{OT1}{rust}{}
%%  \DeclareFontShape{OT1}{rust}{m}{n}{ <-> rust10 }{} 
  \DeclareFontShape{OT1}{rust}{m}{n}{ <-8.5> rust7 <8.5-15> rust10 <15-> rust17 }{} 
  \DeclareFontShape{OT1}{rust}{m}{sl}{ <-> sub * rust/m/n }{}
  \DeclareFontShape{OT1}{rust}{m}{it}{ <-> sub * rust/m/n }{}
  \DeclareFontShape{OT1}{rust}{m}{sc}{ <-> sub * rust/m/n }{}
  \DeclareFontShape{OT1}{rust}{m}{u}{ <-> sub * rust/m/n }{}
%%  \DeclareFontShape{OT1}{rust}{bx}{n}{ <-> rustb10 }{}
  \DeclareFontShape{OT1}{rust}{bx}{n}{ <-8.5> rustb7 <8.5-15> rustb10 <15-> rustb17 }{}
  \DeclareFontShape{OT1}{rust}{bx}{it}{ <-> sub * rust/bx/n }{}
  \DeclareFontShape{OT1}{rust}{bx}{sl}{ <-> sub * rust/bx/n }{}
  \DeclareFontShape{OT1}{rust}{b}{n}{ <-> sub * rust/bx/n }{}
%</fdot1>
%    \end{macrocode}
%
%
%    \begin{macrocode}
%<*fdt1>
\DeclareFontFamily{T1}{rust}{}
  \DeclareFontShape{T1}{rust}{m}{n}{ <-8.5> rust7 <8.5-15> rust10 <15-> rust17 }{}
  \DeclareFontShape{T1}{rust}{m}{sl}{ <-> sub * rust/m/n }{}
  \DeclareFontShape{T1}{rust}{m}{it}{ <-> sub * rust/m/n }{}
  \DeclareFontShape{T1}{rust}{m}{sc}{ <-> sub * rust/m/n }{}
  \DeclareFontShape{T1}{rust}{m}{u}{ <-> sub * rust/m/n }{}
  \DeclareFontShape{T1}{rust}{bx}{n}{ <-8.5> rustb7 <8.5-15> rustb10 <15-> rustb17 }{}
  \DeclareFontShape{T1}{rust}{bx}{it}{ <-> sub * rust/bx/n }{}
  \DeclareFontShape{T1}{rust}{bx}{sl}{ <-> sub * rust/bx/n }{}
  \DeclareFontShape{T1}{rust}{b}{n}{ <-> sub * rust/bx/n }{}
%</fdt1>
%    \end{macrocode}
%
% \section{The \Lpack{rustic} and \Lpack{allrust} package code} \label{sec:code}
%
% \subsection{The \Lpack{rustic} code}
%
%    The \Lpack{rustic} package is intended for use when typesetting examples
% of the book-hand in a document that is otherwise typeset in normal fonts.
%
%    Announce the name and version of the package, which requires
% \LaTeXe{}.
%    \begin{macrocode}
%<*usc>
\NeedsTeXFormat{LaTeX2e}
\ProvidesPackage{rustic}[1999/05/01 v1.0 package for Roman Rustic fonts]
%    \end{macrocode}
%
% \begin{macro}{\Tienc}
%  A  macro for testing the value of |\encodingdefault|.
%    \begin{macrocode}
\providecommand{\Tienc}{T1}
%    \end{macrocode}
% \end{macro}
%
% \begin{macro}{\rustfamily}
%    Selects the Roman Rustic font family in either T1 or OT1 encoding.
%    \begin{macrocode}
\ifx\Tienc\encodingdefault
  \newcommand{\rustfamily}{\usefont{T1}{rust}{m}{n}}
\else
  \newcommand{\rustfamily}{\usefont{OT1}{rust}{m}{n}}
\fi
%    \end{macrocode}
% \end{macro}
%
% \begin{macro}{\textrust}
%    Text command for the Roman Rustic font family.
%    \begin{macrocode}
\DeclareTextFontCommand{\textrust}{\rustfamily}
%    \end{macrocode}
% \end{macro}
%
%    The end of this package.
%    \begin{macrocode}
%</usc>
%    \end{macrocode}
%
% \subsection{The \Lpack{allrust} code}
%
%    The \Lpack{allrust} package is intended for use when the entire document
% will be typeset in the book-hand. This is a minimal package. Because the
% book-hand should not contain arabic digits a fuller package would redefine
% anything numbered in \LaTeX{} to use roman instead of arabic numbering.
% Also, because of the large |\baselineskip| many other aspects of \LaTeX{}
% to do with vertical positioning should also be redefined. 
%
%
%    Announce the name and version of the package, which requires
% \LaTeXe{}. It also uses the \Lpack{rustic} package.
%    \begin{macrocode}
%<*uscall>
\NeedsTeXFormat{LaTeX2e}
\ProvidesPackage{allrust}[1999/05/01 v1.0 package for all Roman Rustic fonts]
\RequirePackage{rustic}
%    \end{macrocode}
%
% \begin{macro}{\Tienc}
%  A  macro for testing the value of |\encodingdefault|.
%    \begin{macrocode}
\providecommand{\Tienc}{T1}
%    \end{macrocode}
% \end{macro}
%
%  Redefine the default fonts to be Roman Rustic, which has only one family 
% member.
% \begin{macro}{\rmdefault}
% \begin{macro}{\sfdefault}
% \begin{macro}{\ttdefault}
%    \begin{macrocode}
\renewcommand{\rmdefault}{rust}
\renewcommand{\sfdefault}{rust}
\renewcommand{\ttdefault}{rust}
%    \end{macrocode}
% \end{macro}
% \end{macro}
% \end{macro}
%
% The `leading' in the book-hand is the same as the |x-height|, which is much
% greater than for the CMR fonts. I have borrowed code from Frank Jensen's
% \Lpack{beton} package~\cite{BETON} to do this.
%
% \begin{macro}{\rustic@baselineskip@table}
% A table of the normal font sizes and the corresponding baselineskip.
%    \begin{macrocode}
\newcommand{\rustic@baselineskip@table}{%
  <\@vpt>9.5%
  <\@vipt>11.4%
  <\@viipt>13.3%
  <\@viiipt>15.2%
  <\@xpt>19%
  <\@xipt>20.9%
  <\@xiipt>22.8%
  <\@xivpt>27.36%
  <\@xviipt>23.3%
  <\@xxpt>38%
  <\@xxvpt>47.5}
%    \end{macrocode}
% \end{macro}
%
% \begin{macro}{\rustic@new@setfontsize}
%  This is a macro that replaces the |\@setfontsize| macro which is called by
% the font size changing commands.
%    \begin{macrocode}
\newcommand{\rustic@new@setfontsize}[3]{%
  \edef\@tempa{\def\noexpand\@tempb####1<#2}%
  \@tempa>##2<##3\@nil{\def\rustic@baselineskip@value{##2}}%
  \edef\@tempa{\noexpand\@tempb\rustic@baselineskip@table<#2}%
  \@tempa><\@nil
  \ifx\rustic@baselineskip@value\@empty
    \def\rustic@baselineskip@value{#3}%
  \fi
  \rustic@old@setfontsize{#1}{#2}\rustic@baselineskip@value}
%    \end{macrocode}
% \end{macro}
%
% We had better give an author a means of using The Computer Modern fonts
% if necessary.
%
% \begin{macro}{\cmrfamily}
% \begin{macro}{\cmssfamily}
% \begin{macro}{\cmttfamily}
%    These macros select the Computer Modern Roman, Sans, and Typewriter 
% font families in either the T1 or OT1 encodings.
%    \begin{macrocode}
\ifx\Tienc\encodingdefault
  \providecommand{\cmrfamily}{\usefont{T1}{cmr}{m}{n}}
  \providecommand{\cmssfamily}{\usefont{T1}{cmss}{m}{n}}
  \providecommand{\cmttfamily}{\usefont{T1}{cmtt}{m}{n}}
\else
  \providecommand{\cmrfamily}{\usefont{OT1}{cmr}{m}{n}}
  \providecommand{\cmssfamily}{\usefont{OT1}{cmss}{m}{n}}
  \providecommand{\cmttfamily}{\usefont{OT1}{cmtt}{m}{n}}
\fi
%    \end{macrocode}
% \end{macro}
% \end{macro}
% \end{macro}
%
% \begin{macro}{\textcmr}
%    Text command for the Computer Modern Roman font family.
%    \begin{macrocode}
\DeclareTextFontCommand{\textcmr}{\cmrfamily}
%    \end{macrocode}
% \end{macro}
%
% \begin{macro}{\textcmss}
%    Text command for the Computer Modern Sans font family.
%    \begin{macrocode}
\DeclareTextFontCommand{\textcmss}{\cmssfamily}
%    \end{macrocode}
% \end{macro}
%
% \begin{macro}{\textcmtt}
%    Text command for the Computer Modern Typewriter font family.
%    \begin{macrocode}
\DeclareTextFontCommand{\textcmtt}{\cmttfamily}
%    \end{macrocode}
% \end{macro}
%
%
% At the start of the document, change the |\@setfontsize| macro and call
% the normal font to implement the change.
%    \begin{macrocode}
\AtBeginDocument{%
  \let\rustic@old@setfontsize=\@setfontsize
  \let\@setfontsize=\rustic@new@setfontsize}
\AtBeginDocument{\normalsize}
%    \end{macrocode}
%
%
%    The end of this package.
%    \begin{macrocode}
%</uscall>
%    \end{macrocode}
%
%
% \bibliographystyle{alpha}
% \renewcommand{\refname}{Bibliography}
% \begin{thebibliography}{GMS94a}
% \addcontentsline{toc}{section}{\refname}
%
% \bibitem[And69]{ANDERSON69}
% Donald M.~Anderson.
% \newblock {\em The Art of Written Forms: The Theory and Practice of Calligraphy}.
% \newblock Holt, Rinehart and Winston, 1969.
%
% \bibitem[Bol95]{BOLOGNA95}
% Giulia Bologna.
% \newblock {\em Illuminated Manuscripts: The Book before Gutenberg}.
% \newblock Crescent Books, 1995.
%
% \bibitem[Day95]{DAY95}
% Lewis F.~Day.
% \newblock {\em Alphabets Old \& New}.
% \newblock (3rd edition originally published by B.~T.~Batsford, 1910) 
% \newblock Senate, 1995.
%
% \bibitem[Dro80]{DROGIN80}
% Marc Drogin.
% \newblock {\em Medieval Calligraphy: Its History and Technique}.
% \newblock Allenheld, Osmun \& Co., 1980.
%
% \bibitem[Dru95]{DRUCKER95}
% Johanna Drucker.
% \newblock {\em The Alphabetic Labyrinth}.
% \newblock Thames \& Hudson, 1995.
%
% \bibitem[Fir93]{FIRMAGE93}
% Richard A.~Firmage.
% \newblock {\em The Alphabet Abecedarium}.
% \newblock David R~Goodine, 1993.
%
%
% \bibitem[Har95]{HARRIS95}
% David Harris.
% \newblock {\em The Art of Calligraphy}.
% \newblock DK Publishing, 1995.
%
% \bibitem[Jen95]{BETON}
% Frank Jensen.
% \newblock {\em The BETON package}.
% \newblock (Available from CTAN in \texttt{macros/latex/contrib/supported}). 
% \newblock 1995.
%
%
% \bibitem[Joh71]{JOHNSTON75}
% Edward Johnston (ed. Heather Child).
% \newblock {\em Formal Penship and Other Papers}.
% \newblock Penthalic, 1971.
%
% \bibitem[Knu87]{CM}
% Donald E.~Knuth.
% \newblock {\em Computer Modern Typefaces}.
% \newblock Addison-Wesley, 1987.
%
% \bibitem[Knu92]{METAFONT}
% Donald E.~Knuth.
% \newblock {\em The METAFONTbook}.
% \newblock Addison-Wesley, 1992.
%
%
% \bibitem[GMS94]{GOOSSENS94}
% Michel Goossens, Frank Mittelbach, and Alexander Samarin.
% \newblock {\em The LaTeX Companion}.
% \newblock Addison-Wesley Publishing Company, 1994.
%
% \bibitem[Tho75]{THOMAS75}
% Alan G.~Thomas.
% \newblock {\em Great Books and Book Collectors}.
% \newblock Weidenfield and Nichoson, 1975.
%
% \bibitem[Wil99a]{DOCMFP}
% Peter R.~Wilson.
% \newblock {\em The DOCMFP Package}.
% \newblock (Available from CTAN in \texttt{macros/latex/contrib/supported}). 
% \newblock 1999.
%
% \bibitem[Wil99b]{ROMANNUM}
% Peter R.~Wilson.
% \newblock {\em The ROMANNUM Package}.
% \newblock (Available from CTAN in \texttt{macros/latex/contrib/supported}). 
% \newblock 1999.
%
% \end{thebibliography}
%
%
% \Finale
% \PrintIndex
%
\endinput

%% \CharacterTable
%%  {Upper-case    \A\B\C\D\E\F\G\H\I\J\K\L\M\N\O\P\Q\R\S\T\U\V\W\X\Y\Z
%%   Lower-case    \a\b\c\d\e\f\g\h\i\j\k\l\m\n\o\p\q\r\s\t\u\v\w\x\y\z
%%   Digits        \0\1\2\3\4\5\6\7\8\9
%%   Exclamation   \!     Double quote  \"     Hash (number) \#
%%   Dollar        \$     Percent       \%     Ampersand     \&
%%   Acute accent  \'     Left paren    \(     Right paren   \)
%%   Asterisk      \*     Plus          \+     Comma         \,
%%   Minus         \-     Point         \.     Solidus       \/
%%   Colon         \:     Semicolon     \;     Less than     \<
%%   Equals        \=     Greater than  \>     Question mark \?
%%   Commercial at \@     Left bracket  \[     Backslash     \\
%%   Right bracket \]     Circumflex    \^     Underscore    \_
%%   Grave accent  \`     Left brace    \{     Vertical bar  \|
%%   Right brace   \}     Tilde         \~}


