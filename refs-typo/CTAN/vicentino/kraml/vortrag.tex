\documentclass[draft,12pt,a4paper,german]{article}
\usepackage[german]{babel}
%Updated to Latex2e by James Kifiger.
\newcommand{\MF}{Metafont}
\title{Die humanistische Kursive von Arrighis Operina bis \MF{} -- ein
typographischer Versuch}
\author{Willibald Kraml}
\usepackage[T1]{fontenc}

\begin{document}\tolerance 300
\maketitle

\section{Die kursive Kanzleischrift im humanistischen Italien}

Im 15. und 16. Jahrhundert waren in Italien vor allem zwei
Handschriftenformen im Gebrauch: die humanistische Minuskelschrift oder
Antiqua, eine eher statisch und klassisch wirkende Schrift, die letztlich
auf die karolingische Minuskel zur"uckgeht, aber von den Humanisten
\glqq irrt"umlich\grqq  f"ur eine schon in der Antike vorhandene (und darum auch
besonders gesch"atzte) Schrift gehalten wurde (und eben darum auch 
\glqq Antiqua\grqq 
genannt wurde). Sie wurde vor allem f"ur sch"on ausgearbeitete Manuskripte
verwendet.  Die zweite "ubliche Schrift war dynamischer, rascher zu schreiben
und dementsprechend die \glqq Alltagsschrift\grqq  der gelehrten Schichten 
(und nur
die konnten "uberhaupt lesen und schreiben).  Sie war unter dem Namen
\glqq Cancellaresca\grqq  oder \glqq Cancellaresca bastarda\grqq , also 
\glqq Kanzleischrift\grqq 
bekannt. Auch \glqq Cancellaresca corsiva\grqq  war ein g"angiger Ausdruck,
der sich
auf die M"oglichkeit bezog, diese Schrift sehr fl"ussig schreiben zu k"onnen.

In der ersten H"alfte des 16. Jh. erschienen einige sogenannte
\glqq Schreibmeisterb"ucher\grqq , in denen die korrekte Art gelehrt wird, diese
Kanzleischrift zu schreiben (was mit der korrekten Art, die Feder zu
schneiden und beim Schreiben zu halten, beginnt).

Das erste (und im Titel des Vortrags schon genannte) ist das Werk
\glqq .......\grqq  des Ludovico degli Arrighi, der auch Ludovico de Henriciis
(?) oder
(nach seiner Herkunft) Vicentino genannt wird.
Er war von Beruf Schreiber und Kalligraph in den Kanzleien des Vatikan.
Sein Buch wurde nach seiner handgeschriebenen Vorlage Seite f"ur Seite in
Holz geschnitten und so gedruckt (nach der Methode der \glqq Blockb"ucher\grqq
, die
schon vor Gutenbergs Erfindung in Gebrauch waren). Dadurch ging nat"urlich
ein wesentlicher Teil der Qualit"at der Handschrift verloren.

Ein Hauptkonkurrent Vicentinos war Giovambattista Palatino, der sein Werk
\glqq ........\grqq  nur kurze Zeit sp"ater zur Publikation brachte. Sein Name
ist den
typographisch Interessierten nat"urlich von der gleichnamigen Schrift Hermann
Zapfs bekannt.

Diese Schriftformen blieben beinahe ohne Ver"anderungen "uber etwa 200 Jahre
in Gebrauch, und zwar nicht nur in Italien, sondern im Lauf der Zeit in fast
ganz Europa. So gibt es entsprechende Lehrb"ucher bzw. Abhandlungen "uber
diese Schrift auch von Erasmus, .....


\section{Das Formeninventar der humanistischen\\Kursiven}

Das Inventar der meisten kursiven Minuskeln l"a"st sich auf ganz wenige
Grundformen zur"uckf"uhren, die z.B. in den Buchstaben f, o und n prototypisch
vorhanden sind. Aus der Grundform des f (und der Langform des s) sind auch
die Ober- und Unterl"angen gebildet. Diese Einheitlichkeit der Formen,
verbunden mit einer ziemlich starren Federhaltung und gleichm"a"sigem
Federdruck f"uhrt zu einer "asthetisch durchaus anspruchsvollen, fast
\glqq kalligraphisch\grqq  zu nennenden Schrift. Der daraus resultierende
Nachteil
ist die nur mittelm"a"sig gute Lesbarkeit, die sich aus dem Mangel an
differenzierten Formen ergibt (gerade in der Zone der Oberl"angen und der
oberen H"alfte der Mittell"angen, die nach modernen Erkenntnissen f"ur die
Lesbarkeit besonders wichtig sind). Auch die H"aufung der
\glqq r"uckw"arts\grqq 
gewandten Unterl"angen ist zwar reizvoll, vermindert aber eher das erzielbare
Lesetempo (anders steht es nat"urlich um das Schreibtempo: die Formen der
Kursive lassen sich ganz besonders rasch schreiben, vor allem im Vergleich
zur Antiqua, die h"aufigeres Absetzen und auch zahlreicheres Verdrehen der
Federstellung bzw. Druckwechsel verlangt).

Auff"allig ist, da"s die Schr"aglage anf"anglich nicht sehr ausgepr"agt oder
auch
gar nicht vorhanden war. Sie war eher eine mehr oder minder zuf"allig
hinzukommende Eigenschaft, die sich aus dem raschen und fl"ussigen Schreiben
ergab, die aber nicht als wesentlich angesehen wurde.

Nun zu einigen Buchstaben (Minuskeln) im Detail:
\begin{itemize}
\item[a] Das a hat eine deutlich andere Form als bei der Antiqua -- es besteht
sozusagen aus den Grundformen des o und des i, wobei allerdings die
Rundung des o in "alterer Zeit stets in eine ann"ahernde Dreiecksform
aufgel"ost ist.

\item[g] Das g kommt in zwei Varianten vor, einer vom a abgeleiteten mit
einer weit ausladenden unteren Schleife, und in der Variante, die in der
Antiqua die gew"ohnliche Form darstellt. Diese Form ist vor allem bei
Ludivico sehr beliebt und weist bei ihm eine spezielle Eigenart auf: es
fehlt das sonst gew"ohnlich vorhandenen \glqq Ohr\grqq .

\item[h] Das h ist in der handgeschriebenen Kursive immer sehr formverwandt
mit dem b, und auch in den Schreibmeisterb"uchern wird dieser Umstand betont:
das h ist ein unten offenes b.

\item[k] Das k weist immer eine Schlaufe auf.

\item[s] Das s begegnet uns in zwei Formen: dam kurzen oder runden s und
dem langen s (das wie das f, aber ohne Querstrich geschrieben wird).
Wichtig (heute nur mehr f"ur die Deutschsprachigen) ist die ss-Ligatur, die
entweder in Form von zwei langen s, oder aber h"aufiger in der Kombination
aus langem und kurzem s vorkommt, eine Ligatur, die im Deutschen noch als
scharfes s fortlebt, leider aber oft f"alschlicherweise als \glqq sz\grqq
bezeichnet
wird, eine Zeichenfolge, die es im Deutschen sprachhistorisch nie gegeben
hat. Die ss-Ligatur (ebenso wie die Langform des s) war "ubrigens auch in der
Antiqua zuhause und in praktisch allen Sprachen, die das Lateinalphabet
verwenden, bis ins neunzehnte Jahrhundert hinein in Gebrauch. Der deutsche
\glqq Alleinanspruch\grqq  auf das "s ist also noch nicht sehr alt! Aus dem
Gesagten
ergibt sich nat"urlich auch, da"s etwa der Usus der Schweizer, bei gewissen
Schrifttypen (z.B. in Schreibmaschinenschriften) kein "s, sondern ein ss zu
schreiben, eigentlich durchaus rechtens ist und konsequent, da in diesen
Schriften auch sonst keine Ligaturen verwendet werden.
\end{itemize}

Neben der ss-Ligatur kommen auch eine ganze Reihe von anderen 
\textbf{Ligaturen} vor, von denen einige im Lauf der Zeit v"ollig
untergegangen sind, andere sich nur teilweise erhalten haben (z.B. st,
sp und so weiter).

Die \textbf{Gro"sbuchstaben} (Majuskeln) stellen ein Problem f"ur sich dar: zum
einen kommen sie in den nicht-deutschen Sprachen vergleichsweise selten vor
und beschr"anken sich auf die Initialen von Namen oder von S"atzen bzw.
Abschnitten. Gerade als Initialen legte man fr"uher gro"sen Wert auf
ausgeschm"uckte Formen, auch wenn die f"ur sich genommen vielleicht kaum mehr
lesbar sind. Die Majuskeln sind sowohl in der Antiqua als auch in der
Cancellaresca ein eigentlich fremdes Element, da sie direkt auf die Formen
der lateinischen Antike zur"uckgehen, w"ahrende die Minuskeln das Ergebnis
einer mehrhundertj"ahrigen kontinuierlichen Entwicklung sind.

Bei der Antiqua ist dies nicht so auff"allig, weil sie sich in ihrer
Entwicklung aus den karolingischen Minuskeln wieder hin zu mehr Statik
ge"andert hat und damit gut zu den ebenfalls statischen Formen der Kapitalis
pa"st. Im Humanismus wurden dann noch die Serifen bei den Minuskeln in der
Art der Serifen der Kapitalis eingef"uhrt, wodurch ein starkes gemeinsames
Formelement vorhanden war.

In der kursiven Kanzleischrift verwendete man zun"achst im Textinneren
ebenfalls die gew"ohnlichen Kapitalisbuchstaben als Gro"sbuchstaben, obwohl
sie dort vielmehr den Charakter eines Fremdk"orpers haben. F"ur die Initialen
am Textanfang begann man mit sehr kunstvollen, ja geradezu manierierten
Formen zu experimentieren, die dann teilweise (in ihren etwas einfacheren
Formen) auch im Textinneren Verwendung fanden. Die Anf"ange der kursiven
Gro"sbuchstaben wirken auf unser Auge mit wenigen Ausnahmen nur grotesk und
unproportioniert. Man vergleiche das Blatt der kursiven Majuskeln im
Lehrbuch des Palatino mit seinem Schriftkunstwerk, das aus lateinischen
Kapitalisbuchstaben besteht, und man wird verstehen, was ich meine.

\section{Die Kursive als Drucktype}

Schon bald begann man, die Kursive auch als Vorlage f"ur Drucktypen zu
verwenden, hatte aber dabei mit einem technischen Problem zu k"ampfen, n"amlich
den "Uberh"angen der Ober- und Unterl"angen. Zum ersten Mal begegnet uns eine
kursive Drucktype in einem bei Aldus Manutius gedruckten Buch im Jahre 1501,
sie stammt von Francesco Griffo. Er zieht sich aus der Aff"are, indem er die
Ober- und Unterl"angen der Antiqua ann"ahert und mit Serifen versieht, nur
f und s behalten ihre geschwungene Form (wobei sie bei den Unterl"angen etwas
\glqq gestutzt\grqq  wird). Griffo verwendet nur f"ur die Minuskeln kursive
Formen, und
zwar in leichter Schr"aglage, als Gro"sbuchstaben werden normale
(geradestehende)  Kapitalisbuchstaben (die Gro"sbuchstaben der Antiqua)
verwendet.

Durch die Ann"aherung an das Formeninventar erreicht Griffo, da"s die
"Uberh"ange nicht mehr allzu h"aufig sind; was er (wahrscheinlich wohl
unbewu"st) auch erreicht, ist eine besserer Lesbarkeit durch gr"o"sere
Differenzierung in den Ober- und Unterl"angen. 

Wichtig ist die Verwendung dieser neuen Drucktype: Sie wird als
eigenst"andige Buchschrift verwendet und nicht etwa als Auszeichnungsschrift
in Erg"anzung zur Antiqua, wie dies heutzutage f"ur gew"ohnlich der Fall ist.
Die Verwendung zur Auszeichnung kommt erst einige Zeit sp"ater in Gebrauch,
wobei sich die Kursive als selbst"andige Buchschrift durchaus noch einige
Zeit halten konnte.  

Die Tendenz der immer gr"o"seren Ann"aherung der Kursiven an das
Formeninventar der Antiqua ist auch in den folgenden Jahrhunderten
zu beobachten und steht nat"urlich mit der neuen Funktion der
Auszeichnungsschrift in Zusammenhang. Ein wichtiger Schritt in diese
Richtung wird dem Wiener Buchdrucker Johann Singriener zugeschrieben, der
1524 zum ersten Mal schr"agstehnde Kapitalisbuchstaben verwendet. Aber noch
um 1550 werden in Frankreich kursive Schriften mit geradestehenden
Gro"sbuchstaben verwendet. Und lange Zeit sp"ater noch werden  dann die
Gro"sbuchstaben zwar schr"aggestellt, allerdings weniger schr"ag als die
korrespondierenden Minuskeln. Der Endpunkt
dieser Entwicklung findet sich in unserem Jahrhundert in der
\glqq schr"aggestellten Antiqua\grqq  (vgl. Knuths \emph{Computer Modern
Slanted}, bei
der ausschlie"slich nur mehr die Antiquaformen vorkommen, allerdings in
Schr"aglage, womit auch das urspr"unglich nur akzidentelle Merkmal zum
dominierenden, ja zum alleinigen Merkmal wurde. Das gleiche finden wir bei
den schr"agen Formen zu den serifenlosen Schriften. So kommt es, da"s heute
die meisten Leute \emph{kursiv} als gleichbedeutend mit \emph{schr"ag} sehen.

Die n"achsten bedeutenden Versuche nach Griffo, die kursive Schreibschrift
auch als Druckschrift einzusetzen, finden wir bei dem uns schon bekannten
Ludovico Vicentino. Zwei oder drei Drucktypen wurden von ihm entworfen und
von dem ber"uhmten Drucker Blado verwendet, wobei er sich bei seiner ersten
Drucktype bem"uht, den urspr"unglichen Duktus der geschriebenen Schrift
m"oglichst getreu beizubehalten. Sp"ater folgt er (wohl auch prim"ar aus
technischen Gr"unden) dem Beispiel Griffos und n"ahert seine Formen der
Antiqua an. Die Schriftschnitte Vicentinos wurden in unserem Jahrhundert
wieder neu entdeckt und sch"atzen gelernt, was hervorragenden
Pers"onlichkeiten wie dem englischen Kalligraphen Edward Johnston und dem
Typographen Stanley Morison, dem Designer der heutzutage ja schon
notorischen \emph{New Times Roman}, oder dem Drucker Hans Mardersteig zu
verdanken ist. Die Lanston Monotype Corporation, deren typographischer Leiter
Stanley Morison war, hat sich dabei besondere Verdienste erworben -- dieselbe
Lanston Monotype Corporation, die heuer aufgrund wirtschaftlicher
Schwierigkeiten zu existieren aufgeh"ort hat.

\section{Die Renaissance der \glqq Chancery\grqq }

Die Wiederentdeckung der alten Kursiven Arrighis war auch der Anfang einer
bescheidenen Renaissance der Kursiven als Buchschrift. Die englische
Rilke-Ausgabe der Weimarer \emph{Cranach Press} wurde in dieser Type gesetzt,
aber auch andere literarische Werke erschienen wieder in Kursiv, so z.B.
Thomas Manns \emph{Der Tod in Venedig}.

In unserem Jahrhundert fand man dann noch eine weitere
Verwendungsm"oglichkeit f"ur die Schriftformen der humanistischen
Kanzleischrift: Mit Schnitten wie Zapfs \emph{Chancery} oder Zapfs {\em
Medici} begann die Renaissance der Kursiven als Zierschrift, f"ur die man
lange Zeit haupts"achlich die englische Schreibschrift (einen sehr sp"aten
Nachfahren der humanistischen Kursive) eingesetzt hatte. Die Verwendung der
typischen gebogenen Oberl"angen in sehr ausgepr"agter Form gibt diesen
Schriften einen sehr kalligraphischen Effekt, allerdings auf Kosten der
Lesbarkeit, so da"s diese Schriften als Buchschriften nur bedingt geeignet
sind.

Ideale Kompromisse, die den Geist der alten Kursive deutlich sichtbar machen
und doch vorwiegend als Buchschrift konzipiert sind, sind in meinen Augen
Zapfs \emph{Palatino Kursiv} und Matthew Carters \emph{Galliard Italic}.

\section{raggedright Die humanistische Kursive und \MF{}}

Das Zeitalter des Computers auch in der Typographie wurde eigentlich schon mit
Matthew Carters Schrift angesprochen: er zeichnete seine Vorlagen, scannte
sie und bearbeitete sie am Computer mit dem Programm \emph{Ikarus} nach.

Mit \MF{} wurde uns von Donald Knuth ein Hilfsmittel gegeben, direkt die
Formen von Schrifttypen als Programm beschreiben zu k"onnen. Jeder, der sich
daran versucht, mit \MF{} Zeichen zu gestalten, wird aber dabei sehr
schnell merken, da"s dies kein ganz einfaches Unterfangen darstellt. Und f"ur
den Anf"anger stellt sich das doppelte Problem, eine durchaus recht komplexe
Programmiersprache lernen zu m"ussen f"ur die Verwendung in einer Sache, von
der er als solcher meistens auch keine Ahnung hat, n"amlich der Formgebung
von Buchstaben.

Genau vor diesem Dilemma stand ich auch vor einem Jahr, als ich begann,
mich mit \MF{} zu besch"aftigen. Bestehende Formen m"oglichst getreu mit
\MF{} \glqq abbilden\grqq  wollte ich nicht (obwohl so ein Vorgehen
wahrscheinlich
empfehlenswerter w"are). Und bei meiner Suche nach Schriftformen, die sich
m"oglichst einfach mit \MF{} gestalten lie"sen (die Programme f"ur die
einzelnen Zeichen sollten kurz und "uberschaubar bleiben, und mit m"oglichst
wenig Sprachelementen auskommen), kam ich schlie"slich auf die
Kursive.

In \MF{} gibt es unter anderem die M"oglichkeit, definierte 
\glqq Federn\grqq  fixer
Gestalt entlang von Pfaden zu f"uhren und so geschriebene Formen nachzuahmen.
Die Einschr"ankungen liegen darin, da"s diese konzeptuellen Federn w"ahrend
eines
Kurvenzuges von fixer Gestalt, Gr"o"se und Lage sein m"ussen. F"ur die Kursive
ist dies kein allzu gro"ser Nachteil, da auch ihre handschriftlichen
Vorbilder mit einer ziemlich starren Federhaltung und mit wenig
Druckunterschieden (was zu unterschiedlichen Strichbreiten f"uhrt)
geschrieben wurden. Diese Methode wurde auch bei den bei \TeX{}
standardm"a"sig vorhandenen kalligraphischen Gro"sbuchstaben von Neenie
Billawalla erfolgreich verwendet.

Zwei Federformen sind in \MF{} schon vordefiniert: elliptische (mit dem
Sonderfall der kreisf"ormigen) und rechteckige (bzw. quadratische). Beide
sind f"ur die kursiven Schriftformen allerdings nur bedingt geeignet.

Die elliptische Feder f"uhrt zwar zu sehr sch"onen, runden Kurven entlang der
Au"senseite von Rundungen, aber An- und Abstriche (und die sind f"ur die
humanistische Kursive sehr charakteristisch) wirken kraftlos und unelegant. 
Die rechteckige Feder wiederum bewirkt, da"s entlang der Au"senseite von
Rundungen Kurventeile auftreten, die absolut gerade sind; durch die an
diesen Stellen auf der Innenseite zwangsl"aufig auftretende Tendenz zu
spitzen Formen wird der Eindruck noch verst"arkt, ja es kommt sogar zu einer
optischen T"auschung, bei der man meint, da"s an dieser Stelle ein Knick nach
innen vorhanden sei! Auch die "Uberg"ange bei den An- und Abstrichen sind nicht
sch"on, es kommt zu Knicken und auch hier zu optischen T"auschungen, als w"are
die Form des Federstrichs leicht konkav.

Neben den vordefinierten Federformen gibt es auch die M"oglichkeit, selber
neue Formen in Form von Polygonz"ugen zu definieren, wobei allerdings die
Einschr"ankung zu beachten ist, da"s die Form allseitig konvex sein mu"s. Ich
begann mit solchen Formen zu experimentieren und kam schlie"slich auf eine
Form, die von der Grundform der rechteckigen Feder abgeleitet ist, bei der aber
alle vier Seiten leicht konvexe Linienz"uge sind. Au"serdem ging ich nicht von
einem Rechteck aus, sondern von einem Trapez, bei dem die Oberseite l"anger
ist als die Unterseite. Damit konnte ich, wie ich glaube, die Nachteile der
elliptischen bzw. der rechteckigen Federformen eliminieren: die Au"senseiten
von Rundungen bleiben rund, und die An- und Abstriche ergeben harmonische,
kontinuierlich an- bzw. abschwellende Formen mit dennoch markantem Abschlu"s.

Die Formen der einzelnen Zeichen wurden ausschlie"slich am Computer
erarbeitet, wobei die Lage von St"utzpunkten, bestimmte Winkellagen der
Kurven oder in vielen F"allen die sog. \emph{tension} ad hoc festgelegt wurde,
bis eine (mich) einigerma"sen befriedigende Form gefunden wurde. Dabei war
ich nicht bem"uht, die "uberlieferten Formen in allen Details nachzuahmen,
vielmehr war ich bestrebt, eine eigene, pers"onliche Schrift im Geiste der
humanistischen Kursive zu gestalten. Ich entschlo"s mich, da"s diese Schrift
den Namen \emph{Vicentino Corsiva} tragen sollte, als Verbeugung vor dem gro"sen
Schriftk"unstler Ludovico Vicentino.

Ich ging bei meinen Experimenten von \emph{plain \MF{}} aus, also nicht
von den Makros der Computer Modern Familie, da diese "uberwiegend dazu
dienen, das Formeninventar klassizistischer Schriften in den Griff zu
bekommen. Au"serdem wollte ich von einem eher minimalistischen Ansatz
ausgehen um zu sehen, was dabei mit \MF{} m"oglich ist (und wo die Grenzen
eines solchen Ansatzes liegen).  

Ich habe mit den so erarbeiteten Formen auch weiter experimentiert und
versucht, ein gewisses Ma"s an \emph{Metaness} einzubauen; daraus entstanden
zum einen auch halbfette Varianten, und auch zwei weitere
\glqq Schnitte\grqq : mit
einer fast kreisf"ormigen Feder geschrieben und unter fast g"anzlicher
Weglassung der An- und Abstriche wirkt die \emph{Vicentino Modern} wie mit
dem Filzstift geschrieben und wesentlich neuzeitlicher; dazu tragen auch
gewisse Varianten in den Formen (etwa bei g und k) bei, die aber immer von
der \glqq klassischen\grqq  Form abgeleitet sind. Als "au"serst experimentelle
Form ist
die \emph{Vicentino Twist} anzusehen, bei der eine elliptische Feder in
Linksdrehung verwendet wird, um leicht nach links geneigte Zeichen zu
schreiben, die aus den Grundformen der \emph{Vicentino Modern} bestehen. Das
Resultat ist eine durch ihr flimmerndes Schriftbild sehr auff"allige (und
schlecht lesbare und darum f"ur l"angeren Text absolut ungeeignete) Schrift.

Ein Problem ist auch die Codierung der Zeichen; es ist kaum sinnvoll, die
Standardeinteilung von  \TeX{} zu verwenden (weil etwa die griechischen
Zeichen darin eigentlich nichts zu suchen haben), au"serdem sind in diesem
Zusammenhang ja derzeit \glqq Umbauarbeiten\grqq  auf einen 8-bit Zeichensatz im
Gange. Au"serdem entschlo"s ich mich, bei den Ligaturen nicht den CM-Fonts zu
folgen: die Entscheidung, welche Ligaturen vorhanden sein sollen, h"angt ganz
entschieden von der Schriftform und nat"urlich auch von der Sprache ab, f"ur
die man eine Schrift gestaltet. So verzichtete ich z.B. auf die Ligaturen
\emph{fl}, \emph{ffi} und \emph{ffl}, hielt aber eine Ligatur \emph{ch} gerade
f"ur die deutsche Sprache f"ur sinnvoll. Au"serdem war es mir ein Bed"urfnis,
wie bei den alten Kursivschriften "ublich auch alternative Zeichenformen
vorzusehen (eigene Formen f"ur den An- oder Auslaut, die Variante des \emph{g},
die alten Ligaturen \emph{ct}, \emph{st} und \emph{sp}, majuskelgro"se Ziffern  
usw.). Derzeit sind die entstandenen Zeichens"atze ein etwas unsch"oner
Kompromi"s: sie folgen im wesentlichen den alten \TeX-Konventionen, lassen
allerdings einige L"ucken (manche davon ungerechtfertigt) und besetzen eine
Position anders (um die ch-Ligatur unterzubringen). Die alternativen Zeichen
sind in einem zweiten Font untergebracht. Dies ist als Provisorium
anzusehen: es ist unbedingt notwendig, f"ur die Umlaute und auch f"ur die
deutschen Anf"uhrungszeichen eigene Zeichen vorzusehen. In der
\glqq endg"ultigen\grqq 
Version sollen daher die Zeichen in ihrer Anordnung weitgehend den neuen
DC-Fonts entsprechen. Obwohl die Schrift also nicht eigentlich als 
\glqq fertig\grqq 
bezeichnet werden kann, m"ochte ich sie dennoch allen Interessierten zur
Verf"ugung stellen und bin f"ur Kritik und Anregungen nat"urlich dankbar.

\section{Literatur (in Auswahl)}
\begin{itemize}
\item -- Scribes and Sources

\item  Alexander Lawson, Anatomy of a Typeface, London 1990.
 
\item  James Moran, Stanley Morison -- His typographic achievement, London 1971.
 
\item  Jan Tschichold, Meisterbuch der Schrift, Ravensburg 1965.
 
\item  Emil Wetzig, Kursiv als Buchschrift, in: Gutenberg-Jahrbuch 1962,
       S. 39--43.

\end{itemize}

\end{document}

