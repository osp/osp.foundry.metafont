% tryfont.tex    Test Humanist Minuscule fonts
% Author: Peter Wilson (CUA) now at peter.r.wilson@boeing.com
%                            (or pandgwilson@earthlink.net) 
% Copyright 2002, 2003 Peter R. Wilson
%
% This work may be distributed and/or modified under the
% conditions of the LaTeX Project Public License, either
% version 1.3 of this license or (at your option) any 
% later version.
% The latest version of the license is in
%    http://www.latex-project.org/lppl.txt
% and version 1.3 or later is part of all distributions of
% LaTeX version 2003/06/01 or later.
%
% This work has the LPPL maintenance status "author-maintained".
%
% This work consists of the files listed in the README file.
%

%\documentclass[12pt]{article}
\documentclass{article}
\usepackage{humanist}

\newcommand{\ABC}{ABCDEFGHIJKL MNOPQRSTUVWXYZ}
\newcommand{\abc}{abcdefghijkl mnopqrs{}tuvwxyz}
\newcommand{\ABCnb}{ABCDEFGHIJKLMNOPQRSTUVWXYZ}
\newcommand{\abcnb}{abcdefghijklmnopqrs{}tuvwxyz}
\newcommand{\punct}{.,;:!?`' () []}
\newcommand{\figs}{0123456789}
\newcommand{\dashes}{- -- ---}
\newcommand{\ligs}{\&{} \ae{} \AE{} ct st}
\newcommand{\sentence}{%
this is an example of the humanist minuscule font. now is the time for all good
men, and women, to come to the aid of the party while the quick brown fox
jumps over the lazy dog.}

\newcommand{\Sentence}{%
This is an example of the Humanist Minuscule font. Now is the time for all good
men, and women, to come to the aid of the party while the quick brown fox
jumps over the lazy dog.}

\newcommand{\esses}{sa sb sc sd se sf sg sh si sj sk sl sm 
                    sn so sp sq sr ss st su sv sw sx sy sz}

\newcommand{\exes}{xa xb xc xd xe xf xg xh xi xj xk xl xm 
                  xn xo xp xq xr xs xt xu xv xw xx xy xz}
          
\newcommand{\jays}{aj bj cj dj ej fj gj hj ij jj kj lj mj 
                   nj oj pj qj rj sj tj uj vj wj xj yj zj}

\newcommand{\dees}{ad bd cd dd ed fd gd hd id jd kd ld md 
                   nd od pd qd rd sd td ud vd wd xd yd zd}

\newcommand{\ares}{ra rb rc rd re rf rg rh ri rj rk rl rm 
                   rn ro rp rq rr rs rt ru rv rw rx ry rz}

\newcommand{\Esses}{SA SB SC SD SE SF SG SH SI SJ SK SL SM 
                    SN SO SP SQ SR SS ST SU SV SW SX SY SZ}

\newcommand{\Exes}{XA XB XC XD XE XF XG XH XI XJ XK XL XM 
                  XN XO XP XQ XR XS XT XU XV XW XX XY XZ}
          
\newcommand{\Jays}{AJ BJ CJ DJ EJ FJ GJ HJ IJ JJ KJ LJ MJ 
                   NJ OJ PJ QJ RJ SJ TJ UJ VJ WJ XJ YJ ZJ}

\newcommand{\Dees}{AD BD CD DD ED FD GD HD ID JD KD LD MD 
                   ND OD PD QD RD SD TD UD VD WD XD YD ZD}

\newcommand{\Ares}{RA RB RC RD RE RF RG RH RI RJ RK RL RM 
                   RN RO RP RQ RR RS RT RU RV RW RX RY RZ}

\title{Try Humanist Minuscule Fonts}
\author{}
\date{}
\begin{document}
\maketitle

    This provides a short test of the characters in the Humanist Minuscule fonts
--- the \verb|hmin| font family. Pen angle is 25 degrees, 5 and 4 nibs for
normal and bold versions.



\begin{center}
The Humanist Minuscule Huge normal font. \\ \par
{\hminfamily\Huge \ABC\\ \abc\\ \punct{} \dashes\\ \ligs{} \figs\\ \par }
\end{center}

\begin{center}
The Humanist Minuscule font in its normal size \\
\texthmin{\ABCnb{} \abcnb{} \punct{} \dashes{} \ligs{} \figs} \\
\end{center}

\begin{center}
The bold minuscule font, the normal minuscule font, and the bold Computer Modern
Roman, all in the normal size \\
\texthmin{\textbf{\abcnb{} \figs{}  \ligs{}}} \\
\texthmin{\abcnb{} \figs{} \ligs{}} \\
\textbf{\abcnb{} \figs{} \ligs{}} \\
\end{center}

\begin{center}
The bold versions, in Huge and tiny sizes. \par
\hminfamily\bfseries
\Huge \abc{} \figs{}  \ligs{} \par
\tiny \abc{} \figs{}  \ligs{} \par
\end{center}

\begin{center}
The font in the tiny size \\ \par
{\hminfamily\tiny \ABC{} \\ \abc\\ \figs\\  \ligs{} \par } 
\end{center}

\begin{center}
    Some built-in ligatures in the normal font \\
\texthmin{``first active brown dog --- but quick \& red fox?''}
\end{center}

%\begin{center}
%    Individual ligatures in the normal font \\
% \texthmin{\esses} \\
% \texthmin{\exes} \\
% \texthmin{\jays} \\
% \texthmin{\dees} \\
% \texthmin{\ares} \\
% \texthmin{\Esses} \\
% \texthmin{\Exes} \\
% \texthmin{\Jays} \\
% \texthmin{\Dees} \\
% \texthmin{\Ares} \\
%\end{center}

{
\hminfamily 
\sentence{}

\Sentence{}
}
    
    This is the end of the test file.

\end{document}